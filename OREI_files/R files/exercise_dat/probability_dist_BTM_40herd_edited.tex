\documentclass[]{article}
\usepackage{lmodern}
\usepackage{amssymb,amsmath}
\usepackage{ifxetex,ifluatex}
\usepackage{fixltx2e} % provides \textsubscript
\ifnum 0\ifxetex 1\fi\ifluatex 1\fi=0 % if pdftex
  \usepackage[T1]{fontenc}
  \usepackage[utf8]{inputenc}
\else % if luatex or xelatex
  \ifxetex
    \usepackage{mathspec}
  \else
    \usepackage{fontspec}
  \fi
  \defaultfontfeatures{Ligatures=TeX,Scale=MatchLowercase}
\fi
% use upquote if available, for straight quotes in verbatim environments
\IfFileExists{upquote.sty}{\usepackage{upquote}}{}
% use microtype if available
\IfFileExists{microtype.sty}{%
\usepackage[]{microtype}
\UseMicrotypeSet[protrusion]{basicmath} % disable protrusion for tt fonts
}{}
\PassOptionsToPackage{hyphens}{url} % url is loaded by hyperref
\usepackage[unicode=true]{hyperref}
\hypersetup{
            pdftitle={Exploration of 40 herd bulk tank milk data},
            pdfauthor={Caitlin Jeffrey},
            pdfborder={0 0 0},
            breaklinks=true}
\urlstyle{same}  % don't use monospace font for urls
\usepackage[margin=1in]{geometry}
\usepackage{color}
\usepackage{fancyvrb}
\newcommand{\VerbBar}{|}
\newcommand{\VERB}{\Verb[commandchars=\\\{\}]}
\DefineVerbatimEnvironment{Highlighting}{Verbatim}{commandchars=\\\{\}}
% Add ',fontsize=\small' for more characters per line
\usepackage{framed}
\definecolor{shadecolor}{RGB}{248,248,248}
\newenvironment{Shaded}{\begin{snugshade}}{\end{snugshade}}
\newcommand{\KeywordTok}[1]{\textcolor[rgb]{0.13,0.29,0.53}{\textbf{#1}}}
\newcommand{\DataTypeTok}[1]{\textcolor[rgb]{0.13,0.29,0.53}{#1}}
\newcommand{\DecValTok}[1]{\textcolor[rgb]{0.00,0.00,0.81}{#1}}
\newcommand{\BaseNTok}[1]{\textcolor[rgb]{0.00,0.00,0.81}{#1}}
\newcommand{\FloatTok}[1]{\textcolor[rgb]{0.00,0.00,0.81}{#1}}
\newcommand{\ConstantTok}[1]{\textcolor[rgb]{0.00,0.00,0.00}{#1}}
\newcommand{\CharTok}[1]{\textcolor[rgb]{0.31,0.60,0.02}{#1}}
\newcommand{\SpecialCharTok}[1]{\textcolor[rgb]{0.00,0.00,0.00}{#1}}
\newcommand{\StringTok}[1]{\textcolor[rgb]{0.31,0.60,0.02}{#1}}
\newcommand{\VerbatimStringTok}[1]{\textcolor[rgb]{0.31,0.60,0.02}{#1}}
\newcommand{\SpecialStringTok}[1]{\textcolor[rgb]{0.31,0.60,0.02}{#1}}
\newcommand{\ImportTok}[1]{#1}
\newcommand{\CommentTok}[1]{\textcolor[rgb]{0.56,0.35,0.01}{\textit{#1}}}
\newcommand{\DocumentationTok}[1]{\textcolor[rgb]{0.56,0.35,0.01}{\textbf{\textit{#1}}}}
\newcommand{\AnnotationTok}[1]{\textcolor[rgb]{0.56,0.35,0.01}{\textbf{\textit{#1}}}}
\newcommand{\CommentVarTok}[1]{\textcolor[rgb]{0.56,0.35,0.01}{\textbf{\textit{#1}}}}
\newcommand{\OtherTok}[1]{\textcolor[rgb]{0.56,0.35,0.01}{#1}}
\newcommand{\FunctionTok}[1]{\textcolor[rgb]{0.00,0.00,0.00}{#1}}
\newcommand{\VariableTok}[1]{\textcolor[rgb]{0.00,0.00,0.00}{#1}}
\newcommand{\ControlFlowTok}[1]{\textcolor[rgb]{0.13,0.29,0.53}{\textbf{#1}}}
\newcommand{\OperatorTok}[1]{\textcolor[rgb]{0.81,0.36,0.00}{\textbf{#1}}}
\newcommand{\BuiltInTok}[1]{#1}
\newcommand{\ExtensionTok}[1]{#1}
\newcommand{\PreprocessorTok}[1]{\textcolor[rgb]{0.56,0.35,0.01}{\textit{#1}}}
\newcommand{\AttributeTok}[1]{\textcolor[rgb]{0.77,0.63,0.00}{#1}}
\newcommand{\RegionMarkerTok}[1]{#1}
\newcommand{\InformationTok}[1]{\textcolor[rgb]{0.56,0.35,0.01}{\textbf{\textit{#1}}}}
\newcommand{\WarningTok}[1]{\textcolor[rgb]{0.56,0.35,0.01}{\textbf{\textit{#1}}}}
\newcommand{\AlertTok}[1]{\textcolor[rgb]{0.94,0.16,0.16}{#1}}
\newcommand{\ErrorTok}[1]{\textcolor[rgb]{0.64,0.00,0.00}{\textbf{#1}}}
\newcommand{\NormalTok}[1]{#1}
\usepackage{graphicx,grffile}
\makeatletter
\def\maxwidth{\ifdim\Gin@nat@width>\linewidth\linewidth\else\Gin@nat@width\fi}
\def\maxheight{\ifdim\Gin@nat@height>\textheight\textheight\else\Gin@nat@height\fi}
\makeatother
% Scale images if necessary, so that they will not overflow the page
% margins by default, and it is still possible to overwrite the defaults
% using explicit options in \includegraphics[width, height, ...]{}
\setkeys{Gin}{width=\maxwidth,height=\maxheight,keepaspectratio}
\IfFileExists{parskip.sty}{%
\usepackage{parskip}
}{% else
\setlength{\parindent}{0pt}
\setlength{\parskip}{6pt plus 2pt minus 1pt}
}
\setlength{\emergencystretch}{3em}  % prevent overfull lines
\providecommand{\tightlist}{%
  \setlength{\itemsep}{0pt}\setlength{\parskip}{0pt}}
\setcounter{secnumdepth}{0}
% Redefines (sub)paragraphs to behave more like sections
\ifx\paragraph\undefined\else
\let\oldparagraph\paragraph
\renewcommand{\paragraph}[1]{\oldparagraph{#1}\mbox{}}
\fi
\ifx\subparagraph\undefined\else
\let\oldsubparagraph\subparagraph
\renewcommand{\subparagraph}[1]{\oldsubparagraph{#1}\mbox{}}
\fi

% set default figure placement to htbp
\makeatletter
\def\fps@figure{htbp}
\makeatother


\title{\emph{Exploration of 40 herd bulk tank milk data}}
\author{Caitlin Jeffrey}
\date{August 6, 2021}

\begin{document}
\maketitle

\begin{Shaded}
\begin{Highlighting}[]
\KeywordTok{library}\NormalTok{(ggplot2)}
\KeywordTok{library}\NormalTok{(MASS)}
\KeywordTok{library}\NormalTok{(ggpubr)}
\end{Highlighting}
\end{Shaded}

\begin{verbatim}
## Warning: package 'ggpubr' was built under R version 4.0.5
\end{verbatim}

\begin{Shaded}
\begin{Highlighting}[]
\CommentTok{# library(geoR)}
\CommentTok{# library(sgeostat)}
\KeywordTok{library}\NormalTok{(dplyr)}
\end{Highlighting}
\end{Shaded}

\begin{verbatim}
## 
## Attaching package: 'dplyr'
\end{verbatim}

\begin{verbatim}
## The following object is masked from 'package:MASS':
## 
##     select
\end{verbatim}

\begin{verbatim}
## The following objects are masked from 'package:stats':
## 
##     filter, lag
\end{verbatim}

\begin{verbatim}
## The following objects are masked from 'package:base':
## 
##     intersect, setdiff, setequal, union
\end{verbatim}

\begin{Shaded}
\begin{Highlighting}[]
\KeywordTok{library}\NormalTok{(tinytex)}
\end{Highlighting}
\end{Shaded}

\begin{verbatim}
## Warning: package 'tinytex' was built under R version 4.0.5
\end{verbatim}

\subsection{\texorpdfstring{\textbf{\emph{Discrete
distributions}}}{Discrete distributions}}\label{discrete-distributions}

\subsubsection{\texorpdfstring{\textbf{Poisson}}{Poisson}}\label{poisson}

\begin{itemize}
\tightlist
\item
  Range: {[}0,infinity{]}
\item
  Parameters: size = number of events, rate = λ
\item
  Interpretation: Distribution of events that occur during a fixed time
  interval or sampling effort with a constant rate of independent
  events; resembles normal with large λ, or exponential with small λ
  (example: over a 1 month period, what's the probability a bird lays a
  particular number of eggs; assumes constant rate of egg laying and
  eggs laid are independent of each other )
\end{itemize}

\subsubsection{\texorpdfstring{\textbf{Binomial}}{Binomial}}\label{binomial}

\begin{itemize}
\tightlist
\item
  Range: {[}0, \# of trials{]}
\item
  Parameters: size= number of trials; p = probability of positive
  outcome
\item
  Interpretation: Distribution of number of successful independent
  dichotomous trials, with constant p (coin tossing, certain probability
  of getting heads)
\end{itemize}

\subsubsection{\texorpdfstring{\textbf{Negative
Binomial}}{Negative Binomial}}\label{negative-binomial}

\begin{itemize}
\tightlist
\item
  Range: {[}0, infinity{]}
\item
  Parameters: size=number of successes; p = probability of success
\item
  Interpretation: Distribution of number of failures in a series of
  independent Bernouli trials, each with p = probability of a success.
  Generates a discrete distribution that is more heterogeneous
  (``overdispersed'') than Poisson; (how many coins we have to toss
  before GETTING 10 heads?)
\end{itemize}

\subsection{\texorpdfstring{\textbf{\emph{Continuous
distributions}}}{Continuous distributions}}\label{continuous-distributions}

\subsubsection{\texorpdfstring{\textbf{Uniform}}{Uniform}}\label{uniform}

\begin{itemize}
\tightlist
\item
  Range: {[}min,max{]}
\item
  Parameters: min = minimum boundary; max = maximum boundary
\item
  Interpretation: Distribution of a value that is equally likely within
  a specified range (any possible value prob is constant between two
  limits; special case of beta distributions)
\end{itemize}

\subsubsection{\texorpdfstring{\textbf{Normal}}{Normal}}\label{normal}

\begin{itemize}
\tightlist
\item
  Range: {[}neg infinity, infinity{]}
\item
  Parameters: mean = central tendency; SD = standard deviation
\item
  Interpretation: Symmetric bell-shaped curve with unbounded tails
\end{itemize}

\subsubsection{\texorpdfstring{\textbf{Gamma}}{Gamma}}\label{gamma}

\begin{itemize}
\tightlist
\item
  Range: {[}0, infinity{]} (doens't go into 0, but continuous)
\item
  Parameters: shape, scale
\item
  Interpretation: mean=shape x scale, variance=shape x scale\^{}2;
  generates a variety of shapes (including normal and exponential) for
  positive continuous variables; (unlike normal, NO NEG)
\end{itemize}

\subsubsection{\texorpdfstring{\textbf{Beta}}{Beta}}\label{beta}

\begin{itemize}
\tightlist
\item
  Range: {[}0,1{]} (can be rescaled to any range by simple
  multiplication and addition)
\item
  Paramters: shape1, shape2
\item
  Interpretation: if shape1 and shape 2 are integers, interpret as a
  coin toss, with shape1 = \# of successes + 1, shape2 = \# of failures
  + 1. Gives distribution of value of p, estimated from data, which can
  range from exponential through uniform through normal (but all are
  bounded). Setting shape1 and shape2 \textless{}1 yields u-shaped
  distributions. (dist of PROB of getting a certain outcome)
\end{itemize}

\begin{Shaded}
\begin{Highlighting}[]
\NormalTok{btm<-}\KeywordTok{read.csv}\NormalTok{(}\StringTok{"BTM_data_combined_7_1_2021.csv"}\NormalTok{, }\DataTypeTok{na.strings=} \StringTok{"."}\NormalTok{, }\DataTypeTok{header =}\NormalTok{ T)}
\KeywordTok{str}\NormalTok{(btm)}
\end{Highlighting}
\end{Shaded}

\begin{verbatim}
## 'data.frame':    21 obs. of  26 variables:
##  $ Sample_ID              : chr  "OH.01" "OH.02" "OH.03" "OH.04" ...
##  $ RAW_CFU_ml_1.100       : int  400 600 3700 100 1200 2300 1000 1500 1200 800 ...
##  $ log10_rawcfu           : num  2.6 2.78 3.57 2 3.08 ...
##  $ PAST_CFU_ml_1.10       : num  10 10 90 9.99 9.99 10 10 1000 50 9.99 ...
##  $ PI__CFU_ml__1.1000     : int  1000 2000 80000 4000 3000 4000 1000000 120000 2000 2000 ...
##  $ Total_Coli__CFU_ml__1.1: num  0.99 11 0.99 0.99 26 0.99 0.99 0.99 0.99 0.99 ...
##  $ E._coli__CFU_ml__1.1   : num  0.99 2 0.99 0.99 0.99 0.99 0.99 0.99 0.99 0.99 ...
##  $ Staph__CFU_ml__1.10    : num  30 30 30 20 50 40 130 40 10 9.99 ...
##  $ perc_BF                : num  3.61 4.26 4.8 5.43 3.71 5.09 3.98 3.91 4.07 3.88 ...
##  $ perc_Protein           : num  2.93 3.52 3.82 3.46 2.92 3.52 3.01 2.95 3.08 2.88 ...
##  $ perc_Lactose           : num  4.6 4.6 4.6 4.63 4.68 4.63 4.65 4.62 4.65 4.56 ...
##  $ perc_Other.Solids      : num  5.7 5.69 5.69 5.69 5.83 5.74 5.73 5.72 5.74 5.65 ...
##  $ MUN                    : num  9.1 9.57 11.7 9.41 13.68 ...
##  $ SCC_1.1000             : int  140000 99000 200000 85000 250000 180000 250000 98000 54000 200000 ...
##  $ log2_SCC               : num  3.49 2.99 4 2.77 4.32 3.85 4.32 2.97 2.11 4 ...
##  $ log10_scc              : num  5.15 5 5.3 4.93 5.4 ...
##  $ Coliforms              : int  0 0 0 0 0 0 0 0 5 0 ...
##  $ log10_coliforms        : num  NA NA NA NA NA ...
##  $ Non_ag_Strep           : int  35 10 30 35 30 205 130 35 35 70 ...
##  $ log10_strep            : num  1.54 1 1.48 1.54 1.48 ...
##  $ Staph_aureus           : int  15 30 0 0 55 0 320 0 0 55 ...
##  $ log10_aureus           : num  1.18 1.48 0 0 1.74 ...
##  $ Staph_sp               : int  0 40 110 65 20 175 15 125 30 665 ...
##  $ log10_staphsp          : num  0 1.6 2.04 1.81 1.3 ...
##  $ Strep_ag               : int  0 0 0 0 0 0 0 0 0 0 ...
##  $ Mycoplasma             : chr  "Neg" "Neg" "Neg" "Neg" ...
\end{verbatim}

\begin{Shaded}
\begin{Highlighting}[]
\NormalTok{btm}\OperatorTok{$}\NormalTok{SCC_}\FloatTok{1.1000}\NormalTok{<-}\KeywordTok{as.numeric}\NormalTok{(btm}\OperatorTok{$}\NormalTok{SCC_}\FloatTok{1.1000}\NormalTok{)}
\NormalTok{btm}\OperatorTok{$}\NormalTok{RAW_CFU_ml_}\FloatTok{1.100}\NormalTok{<-}\KeywordTok{as.numeric}\NormalTok{(btm}\OperatorTok{$}\NormalTok{RAW_CFU_ml_}\FloatTok{1.100}\NormalTok{)}
\NormalTok{btm}\OperatorTok{$}\NormalTok{PI__CFU_ml__}\FloatTok{1.1000}\NormalTok{<-}\KeywordTok{as.numeric}\NormalTok{(btm}\OperatorTok{$}\NormalTok{PI__CFU_ml__}\FloatTok{1.1000}\NormalTok{)}
\NormalTok{btm}\OperatorTok{$}\NormalTok{Coliforms<-}\KeywordTok{as.numeric}\NormalTok{(btm}\OperatorTok{$}\NormalTok{Coliforms)}
\NormalTok{btm}\OperatorTok{$}\NormalTok{Non_ag_Strep<-}\KeywordTok{as.numeric}\NormalTok{(btm}\OperatorTok{$}\NormalTok{Non_ag_Strep)}
\NormalTok{btm}\OperatorTok{$}\NormalTok{Staph_aureus<-}\KeywordTok{as.numeric}\NormalTok{(btm}\OperatorTok{$}\NormalTok{Staph_aureus)}
\NormalTok{btm}\OperatorTok{$}\NormalTok{Staph_sp<-}\KeywordTok{as.numeric}\NormalTok{(btm}\OperatorTok{$}\NormalTok{Staph_sp)}
\CommentTok{# btm$log10_coliforms<-as.numeric(btm$log10_coliforms)}
\CommentTok{# btm$log10_staphsp<-as.numeric(btm$log10_staphsp)}
\CommentTok{# btm$log10_aureus<-as.numeric(btm$log10_aureus)}
\end{Highlighting}
\end{Shaded}

\subsubsection{\texorpdfstring{\textbf{Maximum likelihood estimators of
the
parameters}}{Maximum likelihood estimators of the parameters}}\label{maximum-likelihood-estimators-of-the-parameters}

\begin{itemize}
\tightlist
\item
  MLE's - probability of the data, given the hypothesis (standard would
  be null hypothesis - how probable are data given null is true)
\item
  vs.~what is probability of data given the parameters - how well does
  data fit given parameters of distribution - goodness of fit test
\item
  vs.~what is probability of the parameters, given the data - for a
  given data set, what is probability that certain set of parameters
  would be there- want parameter combinations that make it most likely
  to have come up with the data; maximize this probability (parameters
  given this data)
\end{itemize}

\subsubsection{\texorpdfstring{\textbf{Q-Q and P-P plots to assess
normality}}{Q-Q and P-P plots to assess normality}}\label{q-q-and-p-p-plots-to-assess-normality}

\begin{itemize}
\item
  Probability plot: The data are plotted against a theoretical
  distribution in such a way that the points should form approximately a
  straight line. Departures from this straight line indicate departures
  from the specified distribution. The correlation coefficient
  associated with the linear fit to the data in the probability plot is
  a measure of the goodness of the fit
\item
  The P-P plot plots the cumulative probability of a variable against
  the cumulative probability of a particular distribution (e.g., normal
  distribution). After data are ranked and sorted, the corresponding
  z-score is calculated. This is the expected value that the score
  should have in a normal distribution. The scores are then themselves
  converted to z-scores. The actual z-scores are plotted against the
  expected z-scores. If the data are normally distributed, the result
  would be a straight diagonal line
\item
  Q-Q plot is very similar to the P-P plot except that it plots the
  quantiles (values that split a data set into equal portions) of the
  data set instead of every individual score in the data
\item
  Residuals skewed to the right = normal plot curves below 45-degree
  line (convex); residuals skewed to the left = normal plot curves above
  45-degree line (concave); residuals too peaked = normal plot will be
  sigmoid curved
\end{itemize}

\subsubsection{\texorpdfstring{\textbf{Log
transformation}}{Log transformation}}\label{log-transformation}

\begin{itemize}
\item
  Patel 2019 (bacterial counts): ``Because bedding bacteria counts
  (cfu/cm\^{}3) and BTM culture results (cfu/mL) were not normally
  distributed, results were transformed (log10) before further analysis.
  Continuous variables underwent correlation analysis to identify
  variables that were highly associated (R2 \textgreater{}/= 0.60).
  Simple associations among categorized variables were evaluated using a
  chi-squared test.''
\item
  ``BTSCS is a logarithmic transformation of the BTSCC developed by
  Shook (1982), obtained by the equation: BTSCS = log2(BTSCC/100.000) +
  3''
\item
  Cicconi-Hogan 2013 (\emph{Risk factors associated with bulk tank
  standard plate count, bulk tank coliform count, and the presence of
  Staphylococcus aureus on organic and conventional dairy farms in the
  United States;} standard plate count, coliform count, S. aureus):
  ``Outcome variables analyzed were SPC from the bulk tank, the presence
  of Staph. aureus in the bulk tank, and the CC of the bulk tank milk.
  The presence of Staph. aureus was analyzed as a binary variable (yes,
  no). The distribution of the SPC and CC were both truncated on the
  left at 0 and skewed to the right. The SPC was transformed to log10
  cfu/mL (hereafter, LSPC) and reported as a geometric mean. The CC was
  dichotomized, based on previous research and milk quality
  recommendations (Jayarao et al., 2004; Schroeder, 2009; Elmoslemany et
  al., 2010), at either \textless{}/= 50 cfu/mL (= 0) or
  \textgreater{}50 cfu/mL (= 1).''
\item
  Cicconi-Hogan 2013 (\emph{Associations of risk factors with somatic
  cell count in bulk tank milk on organic and conventional dairy farms
  in the United States;} SCC, SPC, laboratory-pasteurized count, E.
  coli): ``Somatic cell count and SPC were transformed into log10 values
  of the number of cells per milliliter and log10 values of the number
  of colony-forming units per milliliter, respectively, and reported as
  geometric means using the antilog to back-transform the parameters.
  Due to the large number of negative results, LP count and E. coli were
  dichotomized.''
\end{itemize}

\subsection{\texorpdfstring{\textbf{\emph{SCC}}}{SCC}}\label{scc}

\begin{Shaded}
\begin{Highlighting}[]
\KeywordTok{summary}\NormalTok{(btm}\OperatorTok{$}\NormalTok{SCC_}\FloatTok{1.1000}\NormalTok{)}
\end{Highlighting}
\end{Shaded}

\begin{verbatim}
##    Min. 1st Qu.  Median    Mean 3rd Qu.    Max. 
##   54000   99000  140000  144286  180000  250000
\end{verbatim}

\begin{Shaded}
\begin{Highlighting}[]
\NormalTok{scc_data<-btm}\OperatorTok{$}\NormalTok{SCC_}\FloatTok{1.1000}
\NormalTok{z <-}\KeywordTok{fitdistr}\NormalTok{(scc_data,}\StringTok{"normal"}\NormalTok{)}
\KeywordTok{print}\NormalTok{(z) }\CommentTok{# mean and SD, the MLE's (best fit for scc data; most likely if scc data is normally distributed)}
\end{Highlighting}
\end{Shaded}

\begin{verbatim}
##       mean          sd    
##   144285.714    52634.538 
##  ( 11485.798) (  8121.685)
\end{verbatim}

\begin{Shaded}
\begin{Highlighting}[]
\CommentTok{# plot the density function for the normal data with these parameters and annotate with original data points (see smooth density normal curve, and where our data lie on that curve- does model fit data in reasonable way)}

\NormalTok{x <-}\StringTok{ }\DecValTok{0}\OperatorTok{:}\DecValTok{300000} \CommentTok{# range of x values across where you want distribution}
\NormalTok{p_density <-}\StringTok{ }\KeywordTok{dnorm}\NormalTok{(}\DataTypeTok{x=}\NormalTok{x,}
                   \DataTypeTok{mean=}\NormalTok{z}\OperatorTok{$}\NormalTok{estimate[}\StringTok{"mean"}\NormalTok{],}
                   \DataTypeTok{sd=}\NormalTok{z}\OperatorTok{$}\NormalTok{estimate[}\StringTok{"sd"}\NormalTok{]) }\CommentTok{# use mean and sd from actual scc data}
\KeywordTok{qplot}\NormalTok{(x,p_density,}\DataTypeTok{geom=}\StringTok{"line"}\NormalTok{) }\OperatorTok{+}\StringTok{ }\KeywordTok{annotate}\NormalTok{(}\DataTypeTok{geom=}\StringTok{"point"}\NormalTok{,}\DataTypeTok{x=}\NormalTok{scc_data,}\DataTypeTok{y=}\FloatTok{0.0000001}\NormalTok{,}\DataTypeTok{color=}\StringTok{"springgreen4"}\NormalTok{) }\OperatorTok{+}\StringTok{ }\KeywordTok{labs}\NormalTok{(}\DataTypeTok{title=}\StringTok{"SCC data against a normal distribution density curve"}\NormalTok{,}
       \DataTypeTok{x=}\StringTok{"BTM SCC of 21 herds"}\NormalTok{,}
       \DataTypeTok{y=}\StringTok{"Probability density of normal dist."}\NormalTok{) }\CommentTok{# see data points (x values) fall right in middle of probability mass curve}
\end{Highlighting}
\end{Shaded}

\includegraphics{probability_dist_BTM_40herd_edited_files/figure-latex/unnamed-chunk-3-1.pdf}

\begin{Shaded}
\begin{Highlighting}[]
\KeywordTok{ggdensity}\NormalTok{(btm}\OperatorTok{$}\NormalTok{SCC_}\FloatTok{1.1000}\NormalTok{, }
          \DataTypeTok{main =} \StringTok{"Density plot of BTM SCC"}\NormalTok{,}
          \DataTypeTok{xlab =} \StringTok{"SCC of 21 herds"}\NormalTok{)}
\end{Highlighting}
\end{Shaded}

\includegraphics{probability_dist_BTM_40herd_edited_files/figure-latex/unnamed-chunk-3-2.pdf}

\begin{Shaded}
\begin{Highlighting}[]
\KeywordTok{ggqqplot}\NormalTok{(btm}\OperatorTok{$}\NormalTok{SCC_}\FloatTok{1.1000}\NormalTok{)}
\end{Highlighting}
\end{Shaded}

\includegraphics{probability_dist_BTM_40herd_edited_files/figure-latex/unnamed-chunk-3-3.pdf}

\subsection{\texorpdfstring{\textbf{\emph{BTSCS = log2(BTM/100,000) +
3}}}{BTSCS = log2(BTM/100,000) + 3}}\label{btscs-log2btm100000-3}

\begin{Shaded}
\begin{Highlighting}[]
\NormalTok{scclog2_data<-btm}\OperatorTok{$}\NormalTok{log2_SCC}
\NormalTok{z <-}\KeywordTok{fitdistr}\NormalTok{(scclog2_data,}\StringTok{"normal"}\NormalTok{)}
\KeywordTok{print}\NormalTok{(z)}
\end{Highlighting}
\end{Shaded}

\begin{verbatim}
##       mean          sd    
##   3.43047619   0.54635133 
##  (0.11922363) (0.08430384)
\end{verbatim}

\begin{Shaded}
\begin{Highlighting}[]
\NormalTok{x <-}\StringTok{ }\DecValTok{0}\OperatorTok{:}\DecValTok{5} \CommentTok{# range of x values across where you want distribution}
\NormalTok{p_density <-}\StringTok{ }\KeywordTok{dnorm}\NormalTok{(}\DataTypeTok{x=}\NormalTok{x,}
                   \DataTypeTok{mean=}\NormalTok{z}\OperatorTok{$}\NormalTok{estimate[}\StringTok{"mean"}\NormalTok{],}
                   \DataTypeTok{sd=}\NormalTok{z}\OperatorTok{$}\NormalTok{estimate[}\StringTok{"sd"}\NormalTok{]) }\CommentTok{# use mean and sd from actual scc data}
\KeywordTok{qplot}\NormalTok{(x,p_density,}\DataTypeTok{geom=}\StringTok{"line"}\NormalTok{) }\OperatorTok{+}\StringTok{ }\KeywordTok{annotate}\NormalTok{(}\DataTypeTok{geom=}\StringTok{"point"}\NormalTok{,}\DataTypeTok{x=}\NormalTok{scclog2_data,}\DataTypeTok{y=}\FloatTok{0.0000001}\NormalTok{,}\DataTypeTok{color=}\StringTok{"springgreen4"}\NormalTok{) }\OperatorTok{+}\StringTok{ }\KeywordTok{labs}\NormalTok{(}\DataTypeTok{title=}\StringTok{"BTSCS data against a normal distribution density curve"}\NormalTok{,}
       \DataTypeTok{x=}\StringTok{"BTM SCC score (log2) of 21 herds"}\NormalTok{,}
       \DataTypeTok{y=}\StringTok{"Probability density of normal dist."}\NormalTok{) }\CommentTok{# see data points (x values) fall right in middle of probability mass curve}
\end{Highlighting}
\end{Shaded}

\includegraphics{probability_dist_BTM_40herd_edited_files/figure-latex/unnamed-chunk-4-1.pdf}

\begin{Shaded}
\begin{Highlighting}[]
\CommentTok{#do the same for the gamma distribution}
\NormalTok{z <-}\KeywordTok{fitdistr}\NormalTok{(scclog2_data,}\StringTok{"gamma"}\NormalTok{)}
\KeywordTok{print}\NormalTok{(z)}
\end{Highlighting}
\end{Shaded}

\begin{verbatim}
##      shape       rate   
##   36.896503   10.755504 
##  (11.335133) ( 3.326757)
\end{verbatim}

\begin{Shaded}
\begin{Highlighting}[]
\NormalTok{p_density <-}\StringTok{ }\KeywordTok{dgamma}\NormalTok{(}\DataTypeTok{x=}\NormalTok{x,}
                   \DataTypeTok{shape=}\NormalTok{z}\OperatorTok{$}\NormalTok{estimate[}\StringTok{"shape"}\NormalTok{],}
                   \DataTypeTok{rate=}\NormalTok{z}\OperatorTok{$}\NormalTok{estimate[}\StringTok{"rate"}\NormalTok{])}
\KeywordTok{qplot}\NormalTok{(x,p_density,}\DataTypeTok{geom=}\StringTok{"line"}\NormalTok{) }\OperatorTok{+}\StringTok{ }\KeywordTok{annotate}\NormalTok{(}\DataTypeTok{geom=}\StringTok{"point"}\NormalTok{,}\DataTypeTok{x=}\NormalTok{scclog2_data,}\DataTypeTok{y=}\FloatTok{0.001}\NormalTok{, }\DataTypeTok{color=}\StringTok{"springgreen4"}\NormalTok{) }\OperatorTok{+}\StringTok{ }\KeywordTok{labs}\NormalTok{(}\DataTypeTok{title=}\StringTok{"BTSCS data against a gamma distribution density curve"}\NormalTok{,}
       \DataTypeTok{x=}\StringTok{"BTM SCC score (log2) of 21 herds"}\NormalTok{,}
       \DataTypeTok{y=}\StringTok{"Probability density of gamma dist."}\NormalTok{) }\CommentTok{#bound at 0}
\end{Highlighting}
\end{Shaded}

\includegraphics{probability_dist_BTM_40herd_edited_files/figure-latex/unnamed-chunk-4-2.pdf}

\begin{Shaded}
\begin{Highlighting}[]
\KeywordTok{ggdensity}\NormalTok{(btm}\OperatorTok{$}\NormalTok{log2_SCC, }
          \DataTypeTok{main =} \StringTok{"Density plot of BTM SCC score (log2)"}\NormalTok{,}
          \DataTypeTok{xlab =} \StringTok{"BTSCS of 21 herds"}\NormalTok{)}
\end{Highlighting}
\end{Shaded}

\includegraphics{probability_dist_BTM_40herd_edited_files/figure-latex/unnamed-chunk-4-3.pdf}

\begin{Shaded}
\begin{Highlighting}[]
\KeywordTok{ggqqplot}\NormalTok{(btm}\OperatorTok{$}\NormalTok{log2_SCC)}
\end{Highlighting}
\end{Shaded}

\includegraphics{probability_dist_BTM_40herd_edited_files/figure-latex/unnamed-chunk-4-4.pdf}

\subsection{\texorpdfstring{\textbf{\emph{Log10 of
SCC}}}{Log10 of SCC}}\label{log10-of-scc}

\begin{Shaded}
\begin{Highlighting}[]
\NormalTok{scclog10_data<-btm}\OperatorTok{$}\NormalTok{log10_scc}
\NormalTok{z <-}\KeywordTok{fitdistr}\NormalTok{(scclog10_data,}\StringTok{"normal"}\NormalTok{)}
\KeywordTok{print}\NormalTok{(z)}
\end{Highlighting}
\end{Shaded}

\begin{verbatim}
##       mean          sd    
##   5.12923302   0.16467899 
##  (0.03593590) (0.02541052)
\end{verbatim}

\begin{Shaded}
\begin{Highlighting}[]
\NormalTok{x <-}\StringTok{ }\DecValTok{3}\OperatorTok{:}\DecValTok{6} \CommentTok{# range of x values across where you want distribution}
\NormalTok{p_density <-}\StringTok{ }\KeywordTok{dnorm}\NormalTok{(}\DataTypeTok{x=}\NormalTok{x,}
                   \DataTypeTok{mean=}\NormalTok{z}\OperatorTok{$}\NormalTok{estimate[}\StringTok{"mean"}\NormalTok{],}
                   \DataTypeTok{sd=}\NormalTok{z}\OperatorTok{$}\NormalTok{estimate[}\StringTok{"sd"}\NormalTok{]) }\CommentTok{# use mean and sd from actual scc data}
\KeywordTok{qplot}\NormalTok{(x,p_density,}\DataTypeTok{geom=}\StringTok{"line"}\NormalTok{) }\OperatorTok{+}\StringTok{ }\KeywordTok{annotate}\NormalTok{(}\DataTypeTok{geom=}\StringTok{"point"}\NormalTok{,}\DataTypeTok{x=}\NormalTok{scclog10_data,}\DataTypeTok{y=}\FloatTok{0.0000001}\NormalTok{,}\DataTypeTok{color=}\StringTok{"springgreen4"}\NormalTok{) }\OperatorTok{+}\StringTok{ }\KeywordTok{labs}\NormalTok{(}\DataTypeTok{title=}\StringTok{"Log10 SCC data against a normal distribution density curve"}\NormalTok{,}
       \DataTypeTok{x=}\StringTok{"BTM SCC (log10) of 21 herds"}\NormalTok{,}
       \DataTypeTok{y=}\StringTok{"Probability density of normal dist."}\NormalTok{) }\CommentTok{# see data points (x values) fall right in middle of probability mass curve}
\end{Highlighting}
\end{Shaded}

\includegraphics{probability_dist_BTM_40herd_edited_files/figure-latex/unnamed-chunk-5-1.pdf}

\begin{Shaded}
\begin{Highlighting}[]
\CommentTok{#do the same for the gamma distribution}
\NormalTok{z <-}\KeywordTok{fitdistr}\NormalTok{(scclog10_data,}\StringTok{"gamma"}\NormalTok{)}
\KeywordTok{print}\NormalTok{(z)}
\end{Highlighting}
\end{Shaded}

\begin{verbatim}
##      shape       rate   
##   923.93017   180.13028 
##  (284.92255) ( 55.56377)
\end{verbatim}

\begin{Shaded}
\begin{Highlighting}[]
\NormalTok{p_density <-}\StringTok{ }\KeywordTok{dgamma}\NormalTok{(}\DataTypeTok{x=}\NormalTok{x,}
                   \DataTypeTok{shape=}\NormalTok{z}\OperatorTok{$}\NormalTok{estimate[}\StringTok{"shape"}\NormalTok{],}
                   \DataTypeTok{rate=}\NormalTok{z}\OperatorTok{$}\NormalTok{estimate[}\StringTok{"rate"}\NormalTok{])}
\KeywordTok{qplot}\NormalTok{(x,p_density,}\DataTypeTok{geom=}\StringTok{"line"}\NormalTok{) }\OperatorTok{+}\StringTok{ }\KeywordTok{annotate}\NormalTok{(}\DataTypeTok{geom=}\StringTok{"point"}\NormalTok{,}\DataTypeTok{x=}\NormalTok{scclog10_data,}\DataTypeTok{y=}\FloatTok{0.001}\NormalTok{, }\DataTypeTok{color=}\StringTok{"springgreen4"}\NormalTok{) }\OperatorTok{+}\StringTok{ }\KeywordTok{labs}\NormalTok{(}\DataTypeTok{title=}\StringTok{"Log10 SCC data against a gamma distribution density curve"}\NormalTok{,}
       \DataTypeTok{x=}\StringTok{"BTM SCC (log10) of 21 herds"}\NormalTok{,}
       \DataTypeTok{y=}\StringTok{"Probability density of gamma dist."}\NormalTok{) }\CommentTok{#bound at 0}
\end{Highlighting}
\end{Shaded}

\includegraphics{probability_dist_BTM_40herd_edited_files/figure-latex/unnamed-chunk-5-2.pdf}

\begin{Shaded}
\begin{Highlighting}[]
\KeywordTok{ggdensity}\NormalTok{(btm}\OperatorTok{$}\NormalTok{log10_scc, }
          \DataTypeTok{main =} \StringTok{"Density plot of BTM SCC (log10)"}\NormalTok{,}
          \DataTypeTok{xlab =} \StringTok{"BTM SCC log10 of 21 herds"}\NormalTok{)}
\end{Highlighting}
\end{Shaded}

\includegraphics{probability_dist_BTM_40herd_edited_files/figure-latex/unnamed-chunk-5-3.pdf}

\begin{Shaded}
\begin{Highlighting}[]
\KeywordTok{ggqqplot}\NormalTok{(btm}\OperatorTok{$}\NormalTok{log10_scc)}
\end{Highlighting}
\end{Shaded}

\includegraphics{probability_dist_BTM_40herd_edited_files/figure-latex/unnamed-chunk-5-4.pdf}

\subsection{\texorpdfstring{\textbf{\emph{Raw
CFU}}}{Raw CFU}}\label{raw-cfu}

\begin{Shaded}
\begin{Highlighting}[]
\KeywordTok{summary}\NormalTok{(btm}\OperatorTok{$}\NormalTok{RAW_CFU_ml_}\FloatTok{1.100}\NormalTok{)}
\end{Highlighting}
\end{Shaded}

\begin{verbatim}
##    Min. 1st Qu.  Median    Mean 3rd Qu.    Max. 
##     100     600    1000    5467    1700   72000
\end{verbatim}

\begin{Shaded}
\begin{Highlighting}[]
\NormalTok{rawcfu_data<-btm}\OperatorTok{$}\NormalTok{RAW_CFU_ml_}\FloatTok{1.100}
\NormalTok{z <-}\KeywordTok{fitdistr}\NormalTok{(rawcfu_data,}\StringTok{"normal"}\NormalTok{)}
\KeywordTok{print}\NormalTok{(z) }\CommentTok{# mean and SD, the MLE's}
\end{Highlighting}
\end{Shaded}

\begin{verbatim}
##      mean         sd    
##    5466.667   15278.567 
##  ( 3334.057) ( 2357.534)
\end{verbatim}

\begin{Shaded}
\begin{Highlighting}[]
\NormalTok{x <-}\StringTok{ }\DecValTok{0}\OperatorTok{:}\DecValTok{100000} \CommentTok{# range of x values across where you want distribution}
\NormalTok{p_density <-}\StringTok{ }\KeywordTok{dnorm}\NormalTok{(}\DataTypeTok{x=}\NormalTok{x,}
                   \DataTypeTok{mean=}\NormalTok{z}\OperatorTok{$}\NormalTok{estimate[}\StringTok{"mean"}\NormalTok{],}
                   \DataTypeTok{sd=}\NormalTok{z}\OperatorTok{$}\NormalTok{estimate[}\StringTok{"sd"}\NormalTok{]) }\CommentTok{# use mean and sd from actual data}
\KeywordTok{qplot}\NormalTok{(x,p_density,}\DataTypeTok{geom=}\StringTok{"line"}\NormalTok{) }\OperatorTok{+}\StringTok{ }\KeywordTok{annotate}\NormalTok{(}\DataTypeTok{geom=}\StringTok{"point"}\NormalTok{,}\DataTypeTok{x=}\NormalTok{rawcfu_data,}\DataTypeTok{y=}\FloatTok{0.0000001}\NormalTok{,}\DataTypeTok{color=}\StringTok{"deeppink2"}\NormalTok{) }\OperatorTok{+}\StringTok{ }\KeywordTok{labs}\NormalTok{(}\DataTypeTok{title=}\StringTok{"Raw CFU count data against a normal distribution density curve"}\NormalTok{,}
       \DataTypeTok{x=}\StringTok{"BTM Raw CFU count of 21 herds"}\NormalTok{,}
       \DataTypeTok{y=}\StringTok{"Probability density of normal dist."}\NormalTok{) }\CommentTok{# see data points (x values) fall right in middle of probability mass curve}
\end{Highlighting}
\end{Shaded}

\includegraphics{probability_dist_BTM_40herd_edited_files/figure-latex/unnamed-chunk-6-1.pdf}

\begin{Shaded}
\begin{Highlighting}[]
\CommentTok{# # do the same for the gamma distribution}
\CommentTok{# z <-fitdistr(rawcfu_data,"gamma")}
\CommentTok{# print(z)}
\CommentTok{# p_density <- dgamma(x=x,}
\CommentTok{#                    shape=z$estimate["shape"],}
\CommentTok{#                    rate=z$estimate["rate"])}
\CommentTok{# qplot(x,p_density,geom="line") + annotate(geom="point",x=rawcfu_data,y=0.001, color="red") + labs(title="Raw CFU count data against a gamma distribution density curve",}
\CommentTok{#        x="BTM Raw CFU count of 21 herds",}
\CommentTok{#        y="Probability density of gamma dist.") #bound at 0 #bound at 0}

\KeywordTok{ggdensity}\NormalTok{(btm}\OperatorTok{$}\NormalTok{RAW_CFU_ml_}\FloatTok{1.100}\NormalTok{, }
          \DataTypeTok{main =} \StringTok{"Density plot of BTM Raw CFU count"}\NormalTok{,}
          \DataTypeTok{xlab =} \StringTok{"Raw CFU of 21 herds"}\NormalTok{)}
\end{Highlighting}
\end{Shaded}

\includegraphics{probability_dist_BTM_40herd_edited_files/figure-latex/unnamed-chunk-6-2.pdf}

\begin{Shaded}
\begin{Highlighting}[]
\KeywordTok{ggqqplot}\NormalTok{(btm}\OperatorTok{$}\NormalTok{RAW_CFU_ml_}\FloatTok{1.100}\NormalTok{)}
\end{Highlighting}
\end{Shaded}

\includegraphics{probability_dist_BTM_40herd_edited_files/figure-latex/unnamed-chunk-6-3.pdf}

\begin{Shaded}
\begin{Highlighting}[]
\CommentTok{#---------------------------------------------------}

\CommentTok{# remove outlier}
\NormalTok{rawcfu_data_trim<-rawcfu_data[rawcfu_data}\OperatorTok{<}\DecValTok{20000}\NormalTok{]}
\NormalTok{z <-}\KeywordTok{fitdistr}\NormalTok{(rawcfu_data_trim,}\StringTok{"normal"}\NormalTok{)}
\KeywordTok{print}\NormalTok{(z) }\CommentTok{# mean and SD, the MLE's}
\end{Highlighting}
\end{Shaded}

\begin{verbatim}
##      mean         sd    
##   2140.0000   3564.4635 
##  ( 797.0383) ( 563.5912)
\end{verbatim}

\begin{Shaded}
\begin{Highlighting}[]
\NormalTok{x <-}\StringTok{ }\DecValTok{0}\OperatorTok{:}\DecValTok{20000} \CommentTok{# range of x values across where you want distribution}
\NormalTok{p_density <-}\StringTok{ }\KeywordTok{dnorm}\NormalTok{(}\DataTypeTok{x=}\NormalTok{x,}
                   \DataTypeTok{mean=}\NormalTok{z}\OperatorTok{$}\NormalTok{estimate[}\StringTok{"mean"}\NormalTok{],}
                   \DataTypeTok{sd=}\NormalTok{z}\OperatorTok{$}\NormalTok{estimate[}\StringTok{"sd"}\NormalTok{]) }\CommentTok{# use mean and sd from actual data}
\KeywordTok{qplot}\NormalTok{(x,p_density,}\DataTypeTok{geom=}\StringTok{"line"}\NormalTok{) }\OperatorTok{+}\StringTok{ }\KeywordTok{annotate}\NormalTok{(}\DataTypeTok{geom=}\StringTok{"point"}\NormalTok{,}\DataTypeTok{x=}\NormalTok{rawcfu_data_trim,}\DataTypeTok{y=}\FloatTok{0.0000001}\NormalTok{,}\DataTypeTok{color=}\StringTok{"deeppink2"}\NormalTok{) }\OperatorTok{+}\StringTok{ }\KeywordTok{labs}\NormalTok{(}\DataTypeTok{title=}\StringTok{"Raw CFU count data against a normal distribution density curve"}\NormalTok{,}
       \DataTypeTok{x=}\StringTok{"BTM Raw CFU count of 21 herds"}\NormalTok{,}
       \DataTypeTok{y=}\StringTok{"Probability density of normal dist."}\NormalTok{) }\CommentTok{# see data points (x values) fall right in middle of probability mass curve}
\end{Highlighting}
\end{Shaded}

\includegraphics{probability_dist_BTM_40herd_edited_files/figure-latex/unnamed-chunk-6-4.pdf}

\begin{Shaded}
\begin{Highlighting}[]
\KeywordTok{ggqqplot}\NormalTok{(rawcfu_data_trim)}
\end{Highlighting}
\end{Shaded}

\includegraphics{probability_dist_BTM_40herd_edited_files/figure-latex/unnamed-chunk-6-5.pdf}

\subsection{\texorpdfstring{\textbf{\emph{Raw CFU log10
transformation}}}{Raw CFU log10 transformation}}\label{raw-cfu-log10-transformation}

\begin{Shaded}
\begin{Highlighting}[]
\KeywordTok{summary}\NormalTok{(btm}\OperatorTok{$}\NormalTok{log10_rawcfu)}
\end{Highlighting}
\end{Shaded}

\begin{verbatim}
##    Min. 1st Qu.  Median    Mean 3rd Qu.    Max. 
##   2.000   2.778   3.000   3.096   3.230   4.857
\end{verbatim}

\begin{Shaded}
\begin{Highlighting}[]
\NormalTok{rawcfulog_data<-btm}\OperatorTok{$}\NormalTok{log10_rawcfu}
\NormalTok{z <-}\KeywordTok{fitdistr}\NormalTok{(rawcfulog_data,}\StringTok{"normal"}\NormalTok{)}
\KeywordTok{print}\NormalTok{(z) }\CommentTok{# mean and SD, the MLE's}
\end{Highlighting}
\end{Shaded}

\begin{verbatim}
##       mean          sd    
##   3.09562583   0.62488090 
##  (0.13636019) (0.09642122)
\end{verbatim}

\begin{Shaded}
\begin{Highlighting}[]
\NormalTok{x <-}\StringTok{ }\DecValTok{1}\OperatorTok{:}\DecValTok{6} \CommentTok{# range of x values across where you want distribution}
\NormalTok{p_density <-}\StringTok{ }\KeywordTok{dnorm}\NormalTok{(}\DataTypeTok{x=}\NormalTok{x,}
                   \DataTypeTok{mean=}\NormalTok{z}\OperatorTok{$}\NormalTok{estimate[}\StringTok{"mean"}\NormalTok{],}
                   \DataTypeTok{sd=}\NormalTok{z}\OperatorTok{$}\NormalTok{estimate[}\StringTok{"sd"}\NormalTok{]) }\CommentTok{# use mean and sd from actual data}
\KeywordTok{qplot}\NormalTok{(x,p_density,}\DataTypeTok{geom=}\StringTok{"line"}\NormalTok{) }\OperatorTok{+}\StringTok{ }\KeywordTok{annotate}\NormalTok{(}\DataTypeTok{geom=}\StringTok{"point"}\NormalTok{,}\DataTypeTok{x=}\NormalTok{rawcfulog_data,}\DataTypeTok{y=}\FloatTok{0.0000001}\NormalTok{,}\DataTypeTok{color=}\StringTok{"deeppink2"}\NormalTok{) }\OperatorTok{+}\StringTok{ }\KeywordTok{labs}\NormalTok{(}\DataTypeTok{title=}\StringTok{"Raw CFU log10 count data against a normal distribution density curve"}\NormalTok{,}
       \DataTypeTok{x=}\StringTok{"BTM Raw CFU log10 count of 21 herds"}\NormalTok{,}
       \DataTypeTok{y=}\StringTok{"Probability density of normal dist."}\NormalTok{) }\CommentTok{# see data points (x values) fall right in middle of probability mass curve}
\end{Highlighting}
\end{Shaded}

\includegraphics{probability_dist_BTM_40herd_edited_files/figure-latex/unnamed-chunk-7-1.pdf}

\begin{Shaded}
\begin{Highlighting}[]
\CommentTok{# do the same for the gamma distribution}
\NormalTok{z <-}\KeywordTok{fitdistr}\NormalTok{(rawcfulog_data,}\StringTok{"gamma"}\NormalTok{)}
\KeywordTok{print}\NormalTok{(z)}
\end{Highlighting}
\end{Shaded}

\begin{verbatim}
##      shape       rate   
##   26.361137    8.515610 
##  ( 8.084173) ( 2.636442)
\end{verbatim}

\begin{Shaded}
\begin{Highlighting}[]
\NormalTok{p_density <-}\StringTok{ }\KeywordTok{dgamma}\NormalTok{(}\DataTypeTok{x=}\NormalTok{x,}
                   \DataTypeTok{shape=}\NormalTok{z}\OperatorTok{$}\NormalTok{estimate[}\StringTok{"shape"}\NormalTok{],}
                   \DataTypeTok{rate=}\NormalTok{z}\OperatorTok{$}\NormalTok{estimate[}\StringTok{"rate"}\NormalTok{])}
\KeywordTok{qplot}\NormalTok{(x,p_density,}\DataTypeTok{geom=}\StringTok{"line"}\NormalTok{) }\OperatorTok{+}\StringTok{ }\KeywordTok{annotate}\NormalTok{(}\DataTypeTok{geom=}\StringTok{"point"}\NormalTok{,}\DataTypeTok{x=}\NormalTok{rawcfulog_data,}\DataTypeTok{y=}\FloatTok{0.001}\NormalTok{, }\DataTypeTok{color=}\StringTok{"red"}\NormalTok{) }\OperatorTok{+}\StringTok{ }\KeywordTok{labs}\NormalTok{(}\DataTypeTok{title=}\StringTok{"Raw CFU count log10 data against a gamma distribution density curve"}\NormalTok{,}
       \DataTypeTok{x=}\StringTok{"BTM Raw CFU count log10 of 21 herds"}\NormalTok{,}
       \DataTypeTok{y=}\StringTok{"Probability density of gamma dist."}\NormalTok{) }\CommentTok{#bound at 0 #bound at 0}
\end{Highlighting}
\end{Shaded}

\includegraphics{probability_dist_BTM_40herd_edited_files/figure-latex/unnamed-chunk-7-2.pdf}

\begin{Shaded}
\begin{Highlighting}[]
\KeywordTok{ggdensity}\NormalTok{(btm}\OperatorTok{$}\NormalTok{log10_rawcfu, }
          \DataTypeTok{main =} \StringTok{"Density plot of BTM Raw CFU log10 count"}\NormalTok{,}
          \DataTypeTok{xlab =} \StringTok{"Raw CFU of 21 herds"}\NormalTok{)}
\end{Highlighting}
\end{Shaded}

\includegraphics{probability_dist_BTM_40herd_edited_files/figure-latex/unnamed-chunk-7-3.pdf}

\begin{Shaded}
\begin{Highlighting}[]
\KeywordTok{ggqqplot}\NormalTok{(btm}\OperatorTok{$}\NormalTok{log10_rawcfu)}
\end{Highlighting}
\end{Shaded}

\includegraphics{probability_dist_BTM_40herd_edited_files/figure-latex/unnamed-chunk-7-4.pdf}

\subsection{\texorpdfstring{\textbf{\emph{Past
CFU}}}{Past CFU}}\label{past-cfu}

\begin{Shaded}
\begin{Highlighting}[]
\KeywordTok{summary}\NormalTok{(btm}\OperatorTok{$}\NormalTok{PAST_CFU_ml_}\FloatTok{1.10}\NormalTok{)}
\end{Highlighting}
\end{Shaded}

\begin{verbatim}
##    Min. 1st Qu.  Median    Mean 3rd Qu.    Max. 
##    9.99   10.00   20.00   91.90   60.00 1000.00
\end{verbatim}

\begin{Shaded}
\begin{Highlighting}[]
\NormalTok{pastcfu_data<-btm}\OperatorTok{$}\NormalTok{PAST_CFU_ml_}\FloatTok{1.10}
\NormalTok{z <-}\KeywordTok{fitdistr}\NormalTok{(pastcfu_data,}\StringTok{"normal"}\NormalTok{)}
\KeywordTok{print}\NormalTok{(z) }\CommentTok{# mean and SD, the MLE's}
\end{Highlighting}
\end{Shaded}

\begin{verbatim}
##      mean         sd    
##    91.90286   210.41120 
##  ( 45.91549) ( 32.46715)
\end{verbatim}

\begin{Shaded}
\begin{Highlighting}[]
\NormalTok{x <-}\StringTok{ }\DecValTok{0}\OperatorTok{:}\DecValTok{2000} \CommentTok{# range of x values across where you want distribution}
\NormalTok{p_density <-}\StringTok{ }\KeywordTok{dnorm}\NormalTok{(}\DataTypeTok{x=}\NormalTok{x,}
                   \DataTypeTok{mean=}\NormalTok{z}\OperatorTok{$}\NormalTok{estimate[}\StringTok{"mean"}\NormalTok{],}
                   \DataTypeTok{sd=}\NormalTok{z}\OperatorTok{$}\NormalTok{estimate[}\StringTok{"sd"}\NormalTok{]) }\CommentTok{# use mean and sd from actual data}
\KeywordTok{qplot}\NormalTok{(x,p_density,}\DataTypeTok{geom=}\StringTok{"line"}\NormalTok{) }\OperatorTok{+}\StringTok{ }\KeywordTok{annotate}\NormalTok{(}\DataTypeTok{geom=}\StringTok{"point"}\NormalTok{,}\DataTypeTok{x=}\NormalTok{pastcfu_data,}\DataTypeTok{y=}\FloatTok{0.0000001}\NormalTok{,}\DataTypeTok{color=}\StringTok{"deeppink1"}\NormalTok{) }\OperatorTok{+}\StringTok{ }\KeywordTok{labs}\NormalTok{(}\DataTypeTok{title=}\StringTok{"Past. CFU count data against a normal distribution density curve"}\NormalTok{,}
       \DataTypeTok{x=}\StringTok{"BTM Past. CFU count of 21 herds"}\NormalTok{,}
       \DataTypeTok{y=}\StringTok{"Probability density of normal dist."}\NormalTok{) }\CommentTok{# see data points (x values) fall right in middle of probability mass curve}
\end{Highlighting}
\end{Shaded}

\includegraphics{probability_dist_BTM_40herd_edited_files/figure-latex/unnamed-chunk-8-1.pdf}

\begin{Shaded}
\begin{Highlighting}[]
\CommentTok{# do the same for the gamma distribution}
\NormalTok{z <-}\KeywordTok{fitdistr}\NormalTok{(pastcfu_data,}\StringTok{"gamma"}\NormalTok{)}
\KeywordTok{print}\NormalTok{(z)}
\end{Highlighting}
\end{Shaded}

\begin{verbatim}
##       shape         rate    
##   0.555446019   0.006043867 
##  (0.141901507) (0.002306987)
\end{verbatim}

\begin{Shaded}
\begin{Highlighting}[]
\NormalTok{p_density <-}\StringTok{ }\KeywordTok{dgamma}\NormalTok{(}\DataTypeTok{x=}\NormalTok{x,}
                   \DataTypeTok{shape=}\NormalTok{z}\OperatorTok{$}\NormalTok{estimate[}\StringTok{"shape"}\NormalTok{],}
                   \DataTypeTok{rate=}\NormalTok{z}\OperatorTok{$}\NormalTok{estimate[}\StringTok{"rate"}\NormalTok{])}
\KeywordTok{qplot}\NormalTok{(x,p_density,}\DataTypeTok{geom=}\StringTok{"line"}\NormalTok{) }\OperatorTok{+}\StringTok{ }\KeywordTok{annotate}\NormalTok{(}\DataTypeTok{geom=}\StringTok{"point"}\NormalTok{,}\DataTypeTok{x=}\NormalTok{pastcfu_data,}\DataTypeTok{y=}\FloatTok{0.0000001}\NormalTok{, }\DataTypeTok{color=}\StringTok{"deeppink1"}\NormalTok{) }\OperatorTok{+}\StringTok{ }\KeywordTok{labs}\NormalTok{(}\DataTypeTok{title=}\StringTok{"Past. CFU count data against a gamma distribution density curve"}\NormalTok{,}
       \DataTypeTok{x=}\StringTok{"BTM Past. CFU count of 21 herds"}\NormalTok{,}
       \DataTypeTok{y=}\StringTok{"Probability density of gamma dist."}\NormalTok{) }\CommentTok{#bound at 0}
\end{Highlighting}
\end{Shaded}

\includegraphics{probability_dist_BTM_40herd_edited_files/figure-latex/unnamed-chunk-8-2.pdf}

\begin{Shaded}
\begin{Highlighting}[]
\KeywordTok{ggdensity}\NormalTok{(btm}\OperatorTok{$}\NormalTok{PAST_CFU_ml_}\FloatTok{1.10}\NormalTok{, }
          \DataTypeTok{main =} \StringTok{"Density plot of BTM Past. CFU count"}\NormalTok{,}
          \DataTypeTok{xlab =} \StringTok{"Past. CFU of 21 herds"}\NormalTok{)}
\end{Highlighting}
\end{Shaded}

\includegraphics{probability_dist_BTM_40herd_edited_files/figure-latex/unnamed-chunk-8-3.pdf}

\begin{Shaded}
\begin{Highlighting}[]
\KeywordTok{ggqqplot}\NormalTok{(btm}\OperatorTok{$}\NormalTok{PAST_CFU_ml_}\FloatTok{1.10}\NormalTok{)}
\end{Highlighting}
\end{Shaded}

\includegraphics{probability_dist_BTM_40herd_edited_files/figure-latex/unnamed-chunk-8-4.pdf}

\begin{Shaded}
\begin{Highlighting}[]
\CommentTok{#---------------------------------------------------}

\CommentTok{# remove outlier}
\NormalTok{pastcfu_data_trim<-pastcfu_data[pastcfu_data}\OperatorTok{<}\DecValTok{200}\NormalTok{]}
\NormalTok{z <-}\KeywordTok{fitdistr}\NormalTok{(pastcfu_data_trim,}\StringTok{"normal"}\NormalTok{)}
\KeywordTok{print}\NormalTok{(z) }\CommentTok{# mean and SD, the MLE's}
\end{Highlighting}
\end{Shaded}

\begin{verbatim}
##      mean         sd    
##   46.498000   56.505717 
##  (12.635062) ( 8.934338)
\end{verbatim}

\begin{Shaded}
\begin{Highlighting}[]
\NormalTok{x <-}\StringTok{ }\DecValTok{0}\OperatorTok{:}\DecValTok{300} \CommentTok{# range of x values across where you want distribution}
\NormalTok{p_density <-}\StringTok{ }\KeywordTok{dnorm}\NormalTok{(}\DataTypeTok{x=}\NormalTok{x,}
                   \DataTypeTok{mean=}\NormalTok{z}\OperatorTok{$}\NormalTok{estimate[}\StringTok{"mean"}\NormalTok{],}
                   \DataTypeTok{sd=}\NormalTok{z}\OperatorTok{$}\NormalTok{estimate[}\StringTok{"sd"}\NormalTok{]) }\CommentTok{# use mean and sd from actual data}
\KeywordTok{qplot}\NormalTok{(x,p_density,}\DataTypeTok{geom=}\StringTok{"line"}\NormalTok{) }\OperatorTok{+}\StringTok{ }\KeywordTok{annotate}\NormalTok{(}\DataTypeTok{geom=}\StringTok{"point"}\NormalTok{,}\DataTypeTok{x=}\NormalTok{pastcfu_data_trim,}\DataTypeTok{y=}\FloatTok{0.0000001}\NormalTok{,}\DataTypeTok{color=}\StringTok{"deeppink1"}\NormalTok{) }\OperatorTok{+}\StringTok{ }\KeywordTok{labs}\NormalTok{(}\DataTypeTok{title=}\StringTok{"Past. CFU count data against a normal distribution density curve"}\NormalTok{,}
       \DataTypeTok{x=}\StringTok{"BTM Past. CFU count of 21 herds"}\NormalTok{,}
       \DataTypeTok{y=}\StringTok{"Probability density of normal dist."}\NormalTok{) }\CommentTok{# see data points (x values) fall right in middle of probability mass curve}
\end{Highlighting}
\end{Shaded}

\includegraphics{probability_dist_BTM_40herd_edited_files/figure-latex/unnamed-chunk-8-5.pdf}

\begin{Shaded}
\begin{Highlighting}[]
\KeywordTok{ggqqplot}\NormalTok{(pastcfu_data_trim)}
\end{Highlighting}
\end{Shaded}

\includegraphics{probability_dist_BTM_40herd_edited_files/figure-latex/unnamed-chunk-8-6.pdf}

\subsection{\texorpdfstring{\textbf{\emph{PI
CFU}}}{PI CFU}}\label{pi-cfu}

\begin{Shaded}
\begin{Highlighting}[]
\KeywordTok{summary}\NormalTok{(btm}\OperatorTok{$}\NormalTok{PI__CFU_ml__}\FloatTok{1.1000}\NormalTok{)}
\end{Highlighting}
\end{Shaded}

\begin{verbatim}
##    Min. 1st Qu.  Median    Mean 3rd Qu.    Max. 
##     999    2000    4000  173333   94000 1000000
\end{verbatim}

\begin{Shaded}
\begin{Highlighting}[]
\NormalTok{picfu_data<-btm}\OperatorTok{$}\NormalTok{PI__CFU_ml__}\FloatTok{1.1000}
\NormalTok{z <-}\KeywordTok{fitdistr}\NormalTok{(picfu_data,}\StringTok{"normal"}\NormalTok{)}
\KeywordTok{print}\NormalTok{(z) }\CommentTok{# mean and SD, the MLE's}
\end{Highlighting}
\end{Shaded}

\begin{verbatim}
##      mean         sd    
##   173333.29   344269.90 
##  ( 75125.85) ( 53122.00)
\end{verbatim}

\begin{Shaded}
\begin{Highlighting}[]
\NormalTok{x <-}\StringTok{ }\DecValTok{0}\OperatorTok{:}\DecValTok{2000000} \CommentTok{# range of x values across where you want distribution}
\NormalTok{p_density <-}\StringTok{ }\KeywordTok{dnorm}\NormalTok{(}\DataTypeTok{x=}\NormalTok{x,}
                   \DataTypeTok{mean=}\NormalTok{z}\OperatorTok{$}\NormalTok{estimate[}\StringTok{"mean"}\NormalTok{],}
                   \DataTypeTok{sd=}\NormalTok{z}\OperatorTok{$}\NormalTok{estimate[}\StringTok{"sd"}\NormalTok{]) }\CommentTok{# use mean and sd from actual data}
\KeywordTok{qplot}\NormalTok{(x,p_density,}\DataTypeTok{geom=}\StringTok{"line"}\NormalTok{) }\OperatorTok{+}\StringTok{ }\KeywordTok{annotate}\NormalTok{(}\DataTypeTok{geom=}\StringTok{"point"}\NormalTok{,}\DataTypeTok{x=}\NormalTok{picfu_data,}\DataTypeTok{y=}\FloatTok{0.0000001}\NormalTok{,}\DataTypeTok{color=}\StringTok{"deeppink3"}\NormalTok{) }\OperatorTok{+}\StringTok{ }\KeywordTok{labs}\NormalTok{(}\DataTypeTok{title=}\StringTok{"PI CFU count data against a normal distribution density curve"}\NormalTok{,}
       \DataTypeTok{x=}\StringTok{"BTM PI CFU count of 21 herds"}\NormalTok{,}
       \DataTypeTok{y=}\StringTok{"Probability density of normal dist."}\NormalTok{) }\CommentTok{# see data points (x values) fall right in middle of probability mass curve}
\end{Highlighting}
\end{Shaded}

\includegraphics{probability_dist_BTM_40herd_edited_files/figure-latex/unnamed-chunk-9-1.pdf}

\begin{Shaded}
\begin{Highlighting}[]
\CommentTok{# # do the same for the gamma distribution}
\CommentTok{# z <-fitdistr(picfu_data,"gamma")}
\CommentTok{# print(z)}
\CommentTok{# p_density <- dgamma(x=x,}
\CommentTok{#                    shape=z$estimate["shape"],}
\CommentTok{#                    rate=z$estimate["rate"])}
\CommentTok{# qplot(x,p_density,geom="line") + annotate(geom="point",x=picfu_data,y=0.0000001, color="deeppink3") + labs(title="PI CFU count data against a gamma distribution density curve",}
\CommentTok{#        x="BTM PI CFU count of 21 herds",}
\CommentTok{#        y="Probability density of gamma dist.") #bound at 0}

\KeywordTok{ggdensity}\NormalTok{(btm}\OperatorTok{$}\NormalTok{PI__CFU_ml__}\FloatTok{1.1000}\NormalTok{, }
          \DataTypeTok{main =} \StringTok{"Density plot of BTM PI CFU count"}\NormalTok{,}
          \DataTypeTok{xlab =} \StringTok{"PI CFU of 21 herds"}\NormalTok{)}
\end{Highlighting}
\end{Shaded}

\includegraphics{probability_dist_BTM_40herd_edited_files/figure-latex/unnamed-chunk-9-2.pdf}

\begin{Shaded}
\begin{Highlighting}[]
\KeywordTok{ggqqplot}\NormalTok{(btm}\OperatorTok{$}\NormalTok{PI__CFU_ml__}\FloatTok{1.1000}\NormalTok{)}
\end{Highlighting}
\end{Shaded}

\includegraphics{probability_dist_BTM_40herd_edited_files/figure-latex/unnamed-chunk-9-3.pdf}

\subsection{\texorpdfstring{\textbf{\emph{Total
coliforms}}}{Total coliforms}}\label{total-coliforms}

\begin{Shaded}
\begin{Highlighting}[]
\KeywordTok{summary}\NormalTok{(btm}\OperatorTok{$}\NormalTok{Total_Coli__CFU_ml__}\FloatTok{1.1}\NormalTok{)}
\end{Highlighting}
\end{Shaded}

\begin{verbatim}
##    Min. 1st Qu.  Median    Mean 3rd Qu.    Max. 
##   0.990   0.990   0.990   4.565   3.000  26.000
\end{verbatim}

\begin{Shaded}
\begin{Highlighting}[]
\NormalTok{tcoli_data<-btm}\OperatorTok{$}\NormalTok{Total_Coli__CFU_ml__}\FloatTok{1.1}
\NormalTok{z <-}\KeywordTok{fitdistr}\NormalTok{(tcoli_data,}\StringTok{"normal"}\NormalTok{)}
\KeywordTok{print}\NormalTok{(z) }\CommentTok{# mean and SD, the MLE's}
\end{Highlighting}
\end{Shaded}

\begin{verbatim}
##      mean        sd   
##   4.565238   7.224535 
##  (1.576523) (1.114770)
\end{verbatim}

\begin{Shaded}
\begin{Highlighting}[]
\NormalTok{x <-}\StringTok{ }\DecValTok{0}\OperatorTok{:}\DecValTok{30} \CommentTok{# range of x values across where you want distribution}
\NormalTok{p_density <-}\StringTok{ }\KeywordTok{dnorm}\NormalTok{(}\DataTypeTok{x=}\NormalTok{x,}
                   \DataTypeTok{mean=}\NormalTok{z}\OperatorTok{$}\NormalTok{estimate[}\StringTok{"mean"}\NormalTok{],}
                   \DataTypeTok{sd=}\NormalTok{z}\OperatorTok{$}\NormalTok{estimate[}\StringTok{"sd"}\NormalTok{]) }\CommentTok{# use mean and sd from actual data}
\KeywordTok{qplot}\NormalTok{(x,p_density,}\DataTypeTok{geom=}\StringTok{"line"}\NormalTok{) }\OperatorTok{+}\StringTok{ }\KeywordTok{annotate}\NormalTok{(}\DataTypeTok{geom=}\StringTok{"point"}\NormalTok{,}\DataTypeTok{x=}\NormalTok{tcoli_data,}\DataTypeTok{y=}\FloatTok{0.0000001}\NormalTok{,}\DataTypeTok{color=}\StringTok{"deeppink4"}\NormalTok{) }\OperatorTok{+}\StringTok{ }\KeywordTok{labs}\NormalTok{(}\DataTypeTok{title=}\StringTok{"Total coliform count data against a normal distribution density curve"}\NormalTok{,}
       \DataTypeTok{x=}\StringTok{"BTM Total coliform count of 21 herds"}\NormalTok{,}
       \DataTypeTok{y=}\StringTok{"Probability density of normal dist."}\NormalTok{) }\CommentTok{# see data points (x values) fall right in middle of probability mass curve}
\end{Highlighting}
\end{Shaded}

\includegraphics{probability_dist_BTM_40herd_edited_files/figure-latex/unnamed-chunk-10-1.pdf}

\begin{Shaded}
\begin{Highlighting}[]
\CommentTok{# do the same for the gamma distribution}
\NormalTok{z <-}\KeywordTok{fitdistr}\NormalTok{(tcoli_data,}\StringTok{"gamma"}\NormalTok{)}
\KeywordTok{print}\NormalTok{(z)}
\end{Highlighting}
\end{Shaded}

\begin{verbatim}
##      shape       rate   
##   0.7377362   0.1616001 
##  (0.1949484) (0.0592369)
\end{verbatim}

\begin{Shaded}
\begin{Highlighting}[]
\NormalTok{p_density <-}\StringTok{ }\KeywordTok{dgamma}\NormalTok{(}\DataTypeTok{x=}\NormalTok{x,}
                   \DataTypeTok{shape=}\NormalTok{z}\OperatorTok{$}\NormalTok{estimate[}\StringTok{"shape"}\NormalTok{],}
                   \DataTypeTok{rate=}\NormalTok{z}\OperatorTok{$}\NormalTok{estimate[}\StringTok{"rate"}\NormalTok{])}
\KeywordTok{qplot}\NormalTok{(x,p_density,}\DataTypeTok{geom=}\StringTok{"line"}\NormalTok{) }\OperatorTok{+}\StringTok{ }\KeywordTok{annotate}\NormalTok{(}\DataTypeTok{geom=}\StringTok{"point"}\NormalTok{,}\DataTypeTok{x=}\NormalTok{tcoli_data,}\DataTypeTok{y=}\FloatTok{0.0000001}\NormalTok{, }\DataTypeTok{color=}\StringTok{"deeppink4"}\NormalTok{) }\OperatorTok{+}\StringTok{ }\KeywordTok{labs}\NormalTok{(}\DataTypeTok{title=}\StringTok{"Total coliform count data against a gamma distribution density curve"}\NormalTok{,}
       \DataTypeTok{x=}\StringTok{"BTM Total coliform count of 21 herds"}\NormalTok{,}
       \DataTypeTok{y=}\StringTok{"Probability density of gamma dist."}\NormalTok{) }\CommentTok{#bound at 0}
\end{Highlighting}
\end{Shaded}

\includegraphics{probability_dist_BTM_40herd_edited_files/figure-latex/unnamed-chunk-10-2.pdf}

\begin{Shaded}
\begin{Highlighting}[]
\KeywordTok{ggdensity}\NormalTok{(btm}\OperatorTok{$}\NormalTok{Total_Coli__CFU_ml__}\FloatTok{1.1}\NormalTok{, }
          \DataTypeTok{main =} \StringTok{"Density plot of BTM Total coliform count"}\NormalTok{,}
          \DataTypeTok{xlab =} \StringTok{"Total coliform count of 21 herds"}\NormalTok{)}
\end{Highlighting}
\end{Shaded}

\includegraphics{probability_dist_BTM_40herd_edited_files/figure-latex/unnamed-chunk-10-3.pdf}

\begin{Shaded}
\begin{Highlighting}[]
\KeywordTok{ggqqplot}\NormalTok{(btm}\OperatorTok{$}\NormalTok{Total_Coli__CFU_ml__}\FloatTok{1.1}\NormalTok{)}
\end{Highlighting}
\end{Shaded}

\includegraphics{probability_dist_BTM_40herd_edited_files/figure-latex/unnamed-chunk-10-4.pdf}

\subsection{\texorpdfstring{\textbf{\emph{E.
coli}}}{E. coli}}\label{e.-coli}

\begin{Shaded}
\begin{Highlighting}[]
\KeywordTok{summary}\NormalTok{(btm}\OperatorTok{$}\NormalTok{E._coli__CFU_ml__}\FloatTok{1.1}\NormalTok{)}
\end{Highlighting}
\end{Shaded}

\begin{verbatim}
##    Min. 1st Qu.  Median    Mean 3rd Qu.    Max. 
##    0.99    0.99    0.99    1.61    0.99   13.00
\end{verbatim}

\begin{Shaded}
\begin{Highlighting}[]
\NormalTok{ecoli_data<-btm}\OperatorTok{$}\NormalTok{E._coli__CFU_ml__}\FloatTok{1.1}
\NormalTok{z <-}\KeywordTok{fitdistr}\NormalTok{(ecoli_data,}\StringTok{"normal"}\NormalTok{)}
\KeywordTok{print}\NormalTok{(z) }\CommentTok{# mean and SD, the MLE's}
\end{Highlighting}
\end{Shaded}

\begin{verbatim}
##      mean         sd    
##   1.6100000   2.5559250 
##  (0.5577486) (0.3943878)
\end{verbatim}

\begin{Shaded}
\begin{Highlighting}[]
\NormalTok{x <-}\StringTok{ }\DecValTok{0}\OperatorTok{:}\DecValTok{20} \CommentTok{# range of x values across where you want distribution}
\NormalTok{p_density <-}\StringTok{ }\KeywordTok{dnorm}\NormalTok{(}\DataTypeTok{x=}\NormalTok{x,}
                   \DataTypeTok{mean=}\NormalTok{z}\OperatorTok{$}\NormalTok{estimate[}\StringTok{"mean"}\NormalTok{],}
                   \DataTypeTok{sd=}\NormalTok{z}\OperatorTok{$}\NormalTok{estimate[}\StringTok{"sd"}\NormalTok{]) }\CommentTok{# use mean and sd from actual data}
\KeywordTok{qplot}\NormalTok{(x,p_density,}\DataTypeTok{geom=}\StringTok{"line"}\NormalTok{) }\OperatorTok{+}\StringTok{ }\KeywordTok{annotate}\NormalTok{(}\DataTypeTok{geom=}\StringTok{"point"}\NormalTok{,}\DataTypeTok{x=}\NormalTok{ecoli_data,}\DataTypeTok{y=}\FloatTok{0.0000001}\NormalTok{,}\DataTypeTok{color=}\StringTok{"mediumorchid"}\NormalTok{) }\OperatorTok{+}\StringTok{ }\KeywordTok{labs}\NormalTok{(}\DataTypeTok{title=}\StringTok{"E. coli count data against a normal distribution density curve"}\NormalTok{,}
       \DataTypeTok{x=}\StringTok{"BTM E. coli count of 21 herds"}\NormalTok{,}
       \DataTypeTok{y=}\StringTok{"Probability density of normal dist."}\NormalTok{) }\CommentTok{# see data points (x values) fall right in middle of probability mass curve}
\end{Highlighting}
\end{Shaded}

\includegraphics{probability_dist_BTM_40herd_edited_files/figure-latex/unnamed-chunk-11-1.pdf}

\begin{Shaded}
\begin{Highlighting}[]
\CommentTok{# do the same for the gamma distribution}
\NormalTok{z <-}\KeywordTok{fitdistr}\NormalTok{(ecoli_data,}\StringTok{"gamma"}\NormalTok{)}
\KeywordTok{print}\NormalTok{(z)}
\end{Highlighting}
\end{Shaded}

\begin{verbatim}
##      shape       rate   
##   1.6614509   1.0319508 
##  (0.4702745) (0.3403538)
\end{verbatim}

\begin{Shaded}
\begin{Highlighting}[]
\NormalTok{p_density <-}\StringTok{ }\KeywordTok{dgamma}\NormalTok{(}\DataTypeTok{x=}\NormalTok{x,}
                   \DataTypeTok{shape=}\NormalTok{z}\OperatorTok{$}\NormalTok{estimate[}\StringTok{"shape"}\NormalTok{],}
                   \DataTypeTok{rate=}\NormalTok{z}\OperatorTok{$}\NormalTok{estimate[}\StringTok{"rate"}\NormalTok{])}
\KeywordTok{qplot}\NormalTok{(x,p_density,}\DataTypeTok{geom=}\StringTok{"line"}\NormalTok{) }\OperatorTok{+}\StringTok{ }\KeywordTok{annotate}\NormalTok{(}\DataTypeTok{geom=}\StringTok{"point"}\NormalTok{,}\DataTypeTok{x=}\NormalTok{ecoli_data,}\DataTypeTok{y=}\FloatTok{0.0000001}\NormalTok{, }\DataTypeTok{color=}\StringTok{"mediumorchid"}\NormalTok{) }\OperatorTok{+}\StringTok{ }\KeywordTok{labs}\NormalTok{(}\DataTypeTok{title=}\StringTok{"E. coli count data against a gamma distribution density curve"}\NormalTok{,}
       \DataTypeTok{x=}\StringTok{"BTM E. coli count of 21 herds"}\NormalTok{,}
       \DataTypeTok{y=}\StringTok{"Probability density of gamma dist."}\NormalTok{) }\CommentTok{#bound at 0}
\end{Highlighting}
\end{Shaded}

\includegraphics{probability_dist_BTM_40herd_edited_files/figure-latex/unnamed-chunk-11-2.pdf}

\begin{Shaded}
\begin{Highlighting}[]
\KeywordTok{ggdensity}\NormalTok{(btm}\OperatorTok{$}\NormalTok{Total_Coli__CFU_ml__}\FloatTok{1.1}\NormalTok{, }
          \DataTypeTok{main =} \StringTok{"Density plot of BTM E. coli count"}\NormalTok{,}
          \DataTypeTok{xlab =} \StringTok{"E. coli count of 21 herds"}\NormalTok{)}
\end{Highlighting}
\end{Shaded}

\includegraphics{probability_dist_BTM_40herd_edited_files/figure-latex/unnamed-chunk-11-3.pdf}

\begin{Shaded}
\begin{Highlighting}[]
\KeywordTok{ggqqplot}\NormalTok{(btm}\OperatorTok{$}\NormalTok{Total_Coli__CFU_ml__}\FloatTok{1.1}\NormalTok{)}
\end{Highlighting}
\end{Shaded}

\includegraphics{probability_dist_BTM_40herd_edited_files/figure-latex/unnamed-chunk-11-4.pdf}

\subsection{\texorpdfstring{\textbf{\emph{Staph}}}{Staph}}\label{staph}

\begin{Shaded}
\begin{Highlighting}[]
\KeywordTok{summary}\NormalTok{(btm}\OperatorTok{$}\NormalTok{Staph__CFU_ml__}\FloatTok{1.10}\NormalTok{)}
\end{Highlighting}
\end{Shaded}

\begin{verbatim}
##    Min. 1st Qu.  Median    Mean 3rd Qu.    Max. 
##    9.99   10.00   30.00   32.86   40.00  130.00
\end{verbatim}

\begin{Shaded}
\begin{Highlighting}[]
\NormalTok{staph_data<-btm}\OperatorTok{$}\NormalTok{Staph__CFU_ml__}\FloatTok{1.10}
\NormalTok{z <-}\KeywordTok{fitdistr}\NormalTok{(staph_data,}\StringTok{"normal"}\NormalTok{)}
\KeywordTok{print}\NormalTok{(z) }\CommentTok{# mean and SD, the MLE's}
\end{Highlighting}
\end{Shaded}

\begin{verbatim}
##      mean         sd    
##   32.855238   26.213773 
##  ( 5.720314) ( 4.044873)
\end{verbatim}

\begin{Shaded}
\begin{Highlighting}[]
\NormalTok{x <-}\StringTok{ }\DecValTok{0}\OperatorTok{:}\DecValTok{200} \CommentTok{# range of x values across where you want distribution}
\NormalTok{p_density <-}\StringTok{ }\KeywordTok{dnorm}\NormalTok{(}\DataTypeTok{x=}\NormalTok{x,}
                   \DataTypeTok{mean=}\NormalTok{z}\OperatorTok{$}\NormalTok{estimate[}\StringTok{"mean"}\NormalTok{],}
                   \DataTypeTok{sd=}\NormalTok{z}\OperatorTok{$}\NormalTok{estimate[}\StringTok{"sd"}\NormalTok{]) }\CommentTok{# use mean and sd from actual data}
\KeywordTok{qplot}\NormalTok{(x,p_density,}\DataTypeTok{geom=}\StringTok{"line"}\NormalTok{) }\OperatorTok{+}\StringTok{ }\KeywordTok{annotate}\NormalTok{(}\DataTypeTok{geom=}\StringTok{"point"}\NormalTok{,}\DataTypeTok{x=}\NormalTok{staph_data,}\DataTypeTok{y=}\FloatTok{0.0000001}\NormalTok{,}\DataTypeTok{color=}\StringTok{"mediumorchid2"}\NormalTok{) }\OperatorTok{+}\StringTok{ }\KeywordTok{labs}\NormalTok{(}\DataTypeTok{title=}\StringTok{"Staph. count data against a normal distribution density curve"}\NormalTok{,}
       \DataTypeTok{x=}\StringTok{"BTM Staph. count of 21 herds"}\NormalTok{,}
       \DataTypeTok{y=}\StringTok{"Probability density of normal dist."}\NormalTok{) }\CommentTok{# see data points (x values) fall right in middle of probability mass curve}
\end{Highlighting}
\end{Shaded}

\includegraphics{probability_dist_BTM_40herd_edited_files/figure-latex/unnamed-chunk-12-1.pdf}

\begin{Shaded}
\begin{Highlighting}[]
\CommentTok{# do the same for the gamma distribution}
\NormalTok{z <-}\KeywordTok{fitdistr}\NormalTok{(staph_data,}\StringTok{"gamma"}\NormalTok{)}
\KeywordTok{print}\NormalTok{(z)}
\end{Highlighting}
\end{Shaded}

\begin{verbatim}
##      shape         rate   
##   2.11381530   0.06433064 
##  (0.60772643) (0.02086002)
\end{verbatim}

\begin{Shaded}
\begin{Highlighting}[]
\NormalTok{p_density <-}\StringTok{ }\KeywordTok{dgamma}\NormalTok{(}\DataTypeTok{x=}\NormalTok{x,}
                   \DataTypeTok{shape=}\NormalTok{z}\OperatorTok{$}\NormalTok{estimate[}\StringTok{"shape"}\NormalTok{],}
                   \DataTypeTok{rate=}\NormalTok{z}\OperatorTok{$}\NormalTok{estimate[}\StringTok{"rate"}\NormalTok{])}
\KeywordTok{qplot}\NormalTok{(x,p_density,}\DataTypeTok{geom=}\StringTok{"line"}\NormalTok{) }\OperatorTok{+}\StringTok{ }\KeywordTok{annotate}\NormalTok{(}\DataTypeTok{geom=}\StringTok{"point"}\NormalTok{,}\DataTypeTok{x=}\NormalTok{staph_data,}\DataTypeTok{y=}\FloatTok{0.0000001}\NormalTok{, }\DataTypeTok{color=}\StringTok{"mediumorchid2"}\NormalTok{) }\OperatorTok{+}\StringTok{ }\KeywordTok{labs}\NormalTok{(}\DataTypeTok{title=}\StringTok{"Staph. count data against a gamma distribution density curve"}\NormalTok{,}
       \DataTypeTok{x=}\StringTok{"BTM Staph. count of 21 herds"}\NormalTok{,}
       \DataTypeTok{y=}\StringTok{"Probability density of gamma dist."}\NormalTok{) }\CommentTok{#bound at 0}
\end{Highlighting}
\end{Shaded}

\includegraphics{probability_dist_BTM_40herd_edited_files/figure-latex/unnamed-chunk-12-2.pdf}

\begin{Shaded}
\begin{Highlighting}[]
\KeywordTok{ggdensity}\NormalTok{(btm}\OperatorTok{$}\NormalTok{Staph__CFU_ml__}\FloatTok{1.10}\NormalTok{, }
          \DataTypeTok{main =} \StringTok{"Density plot of BTM Staph count (St. Albans)"}\NormalTok{,}
          \DataTypeTok{xlab =} \StringTok{"Staph count of 21 herds"}\NormalTok{)}
\end{Highlighting}
\end{Shaded}

\includegraphics{probability_dist_BTM_40herd_edited_files/figure-latex/unnamed-chunk-12-3.pdf}

\begin{Shaded}
\begin{Highlighting}[]
\KeywordTok{ggqqplot}\NormalTok{(btm}\OperatorTok{$}\NormalTok{Staph__CFU_ml__}\FloatTok{1.10}\NormalTok{)}
\end{Highlighting}
\end{Shaded}

\includegraphics{probability_dist_BTM_40herd_edited_files/figure-latex/unnamed-chunk-12-4.pdf}

\begin{Shaded}
\begin{Highlighting}[]
\CommentTok{#---------------------------------------------------}

\CommentTok{# remove outlier}
\NormalTok{staph_data_trim<-staph_data[staph_data}\OperatorTok{<}\DecValTok{100}\NormalTok{]}
\NormalTok{z <-}\KeywordTok{fitdistr}\NormalTok{(staph_data_trim,}\StringTok{"normal"}\NormalTok{)}
\KeywordTok{print}\NormalTok{(z) }\CommentTok{# mean and SD, the MLE's}
\end{Highlighting}
\end{Shaded}

\begin{verbatim}
##      mean         sd    
##   27.998000   15.035691 
##  ( 3.362083) ( 2.377352)
\end{verbatim}

\begin{Shaded}
\begin{Highlighting}[]
\NormalTok{x <-}\StringTok{ }\DecValTok{0}\OperatorTok{:}\DecValTok{100} \CommentTok{# range of x values across where you want distribution}
\NormalTok{p_density <-}\StringTok{ }\KeywordTok{dnorm}\NormalTok{(}\DataTypeTok{x=}\NormalTok{x,}
                   \DataTypeTok{mean=}\NormalTok{z}\OperatorTok{$}\NormalTok{estimate[}\StringTok{"mean"}\NormalTok{],}
                   \DataTypeTok{sd=}\NormalTok{z}\OperatorTok{$}\NormalTok{estimate[}\StringTok{"sd"}\NormalTok{]) }\CommentTok{# use mean and sd from actual data}
\KeywordTok{qplot}\NormalTok{(x,p_density,}\DataTypeTok{geom=}\StringTok{"line"}\NormalTok{) }\OperatorTok{+}\StringTok{ }\KeywordTok{annotate}\NormalTok{(}\DataTypeTok{geom=}\StringTok{"point"}\NormalTok{,}\DataTypeTok{x=}\NormalTok{staph_data_trim, }\DataTypeTok{y=}\FloatTok{0.0000001}\NormalTok{,}\DataTypeTok{color=}\StringTok{"mediumorchid2"}\NormalTok{) }\OperatorTok{+}\StringTok{ }\KeywordTok{labs}\NormalTok{(}\DataTypeTok{title=}\StringTok{"Staph. count data against a normal distribution density curve"}\NormalTok{,}
       \DataTypeTok{x=}\StringTok{"BTM Staph. count of 21 herds"}\NormalTok{,}
       \DataTypeTok{y=}\StringTok{"Probability density of normal dist."}\NormalTok{) }\CommentTok{# see data points (x values) fall right in middle of probability mass curve}
\end{Highlighting}
\end{Shaded}

\includegraphics{probability_dist_BTM_40herd_edited_files/figure-latex/unnamed-chunk-12-5.pdf}

\begin{Shaded}
\begin{Highlighting}[]
\KeywordTok{ggqqplot}\NormalTok{(staph_data_trim)}
\end{Highlighting}
\end{Shaded}

\includegraphics{probability_dist_BTM_40herd_edited_files/figure-latex/unnamed-chunk-12-6.pdf}

\subsection{\texorpdfstring{\textbf{\emph{Percent
butterfat}}}{Percent butterfat}}\label{percent-butterfat}

\begin{Shaded}
\begin{Highlighting}[]
\KeywordTok{summary}\NormalTok{(btm}\OperatorTok{$}\NormalTok{perc_BF)}
\end{Highlighting}
\end{Shaded}

\begin{verbatim}
##    Min. 1st Qu.  Median    Mean 3rd Qu.    Max. 
##   3.610   3.910   4.260   4.358   4.850   5.430
\end{verbatim}

\begin{Shaded}
\begin{Highlighting}[]
\NormalTok{bf_data<-btm}\OperatorTok{$}\NormalTok{perc_BF}
\NormalTok{z <-}\KeywordTok{fitdistr}\NormalTok{(bf_data,}\StringTok{"normal"}\NormalTok{)}
\KeywordTok{print}\NormalTok{(z) }\CommentTok{# mean and SD, the MLE's}
\end{Highlighting}
\end{Shaded}

\begin{verbatim}
##       mean          sd    
##   4.35809524   0.53713188 
##  (0.11721179) (0.08288125)
\end{verbatim}

\begin{Shaded}
\begin{Highlighting}[]
\NormalTok{x <-}\StringTok{ }\DecValTok{0}\OperatorTok{:}\DecValTok{10} \CommentTok{# range of x values across where you want distribution}
\NormalTok{p_density <-}\StringTok{ }\KeywordTok{dnorm}\NormalTok{(}\DataTypeTok{x=}\NormalTok{x,}
                   \DataTypeTok{mean=}\NormalTok{z}\OperatorTok{$}\NormalTok{estimate[}\StringTok{"mean"}\NormalTok{],}
                   \DataTypeTok{sd=}\NormalTok{z}\OperatorTok{$}\NormalTok{estimate[}\StringTok{"sd"}\NormalTok{]) }\CommentTok{# use mean and sd from actual data}
\KeywordTok{qplot}\NormalTok{(x,p_density,}\DataTypeTok{geom=}\StringTok{"line"}\NormalTok{) }\OperatorTok{+}\StringTok{ }\KeywordTok{annotate}\NormalTok{(}\DataTypeTok{geom=}\StringTok{"point"}\NormalTok{,}\DataTypeTok{x=}\NormalTok{bf_data,}\DataTypeTok{y=}\FloatTok{0.0000001}\NormalTok{,}\DataTypeTok{color=}\StringTok{"royalblue1"}\NormalTok{) }\OperatorTok{+}\StringTok{ }\KeywordTok{labs}\NormalTok{(}\DataTypeTok{title=}\StringTok{"Percent butterfat data against a normal distribution density curve"}\NormalTok{,}
       \DataTypeTok{x=}\StringTok{"BTM Percent butterfat of 21 herds"}\NormalTok{,}
       \DataTypeTok{y=}\StringTok{"Probability density of normal dist."}\NormalTok{) }\CommentTok{# see data points (x values) fall right in middle of probability mass curve}
\end{Highlighting}
\end{Shaded}

\includegraphics{probability_dist_BTM_40herd_edited_files/figure-latex/unnamed-chunk-13-1.pdf}

\begin{Shaded}
\begin{Highlighting}[]
\CommentTok{# do the same for the gamma distribution}
\NormalTok{z <-}\KeywordTok{fitdistr}\NormalTok{(bf_data,}\StringTok{"gamma"}\NormalTok{)}
\KeywordTok{print}\NormalTok{(z)}
\end{Highlighting}
\end{Shaded}

\begin{verbatim}
##      shape       rate   
##   66.967180   15.366160 
##  (20.614475) ( 4.747869)
\end{verbatim}

\begin{Shaded}
\begin{Highlighting}[]
\NormalTok{p_density <-}\StringTok{ }\KeywordTok{dgamma}\NormalTok{(}\DataTypeTok{x=}\NormalTok{x,}
                   \DataTypeTok{shape=}\NormalTok{z}\OperatorTok{$}\NormalTok{estimate[}\StringTok{"shape"}\NormalTok{],}
                   \DataTypeTok{rate=}\NormalTok{z}\OperatorTok{$}\NormalTok{estimate[}\StringTok{"rate"}\NormalTok{])}
\KeywordTok{qplot}\NormalTok{(x,p_density,}\DataTypeTok{geom=}\StringTok{"line"}\NormalTok{) }\OperatorTok{+}\StringTok{ }\KeywordTok{annotate}\NormalTok{(}\DataTypeTok{geom=}\StringTok{"point"}\NormalTok{,}\DataTypeTok{x=}\NormalTok{bf_data,}\DataTypeTok{y=}\FloatTok{0.0000001}\NormalTok{, }\DataTypeTok{color=}\StringTok{"royalblue1"}\NormalTok{) }\OperatorTok{+}\StringTok{ }\KeywordTok{labs}\NormalTok{(}\DataTypeTok{title=}\StringTok{"Percent butterfat data against a gamma distribution density curve"}\NormalTok{,}
       \DataTypeTok{x=}\StringTok{"BTM Percent butterfat of 21 herds"}\NormalTok{,}
       \DataTypeTok{y=}\StringTok{"Probability density of gamma dist."}\NormalTok{) }\CommentTok{#bound at 0}
\end{Highlighting}
\end{Shaded}

\includegraphics{probability_dist_BTM_40herd_edited_files/figure-latex/unnamed-chunk-13-2.pdf}

\begin{Shaded}
\begin{Highlighting}[]
\KeywordTok{ggdensity}\NormalTok{(btm}\OperatorTok{$}\NormalTok{perc_BF, }
          \DataTypeTok{main =} \StringTok{"Density plot of BTM % Butterfat"}\NormalTok{,}
          \DataTypeTok{xlab =} \StringTok{"% Butterfat of 21 herds"}\NormalTok{)}
\end{Highlighting}
\end{Shaded}

\includegraphics{probability_dist_BTM_40herd_edited_files/figure-latex/unnamed-chunk-13-3.pdf}

\begin{Shaded}
\begin{Highlighting}[]
\KeywordTok{ggqqplot}\NormalTok{(btm}\OperatorTok{$}\NormalTok{perc_BF)}
\end{Highlighting}
\end{Shaded}

\includegraphics{probability_dist_BTM_40herd_edited_files/figure-latex/unnamed-chunk-13-4.pdf}

\subsection{\texorpdfstring{\textbf{\emph{Percent
protein}}}{Percent protein}}\label{percent-protein}

\begin{Shaded}
\begin{Highlighting}[]
\KeywordTok{summary}\NormalTok{(btm}\OperatorTok{$}\NormalTok{perc_Protein)}
\end{Highlighting}
\end{Shaded}

\begin{verbatim}
##    Min. 1st Qu.  Median    Mean 3rd Qu.    Max. 
##   2.880   2.950   3.150   3.264   3.520   3.820
\end{verbatim}

\begin{Shaded}
\begin{Highlighting}[]
\NormalTok{pro_data<-btm}\OperatorTok{$}\NormalTok{perc_Protein}
\NormalTok{z <-}\KeywordTok{fitdistr}\NormalTok{(pro_data,}\StringTok{"normal"}\NormalTok{)}
\KeywordTok{print}\NormalTok{(z) }\CommentTok{# mean and SD, the MLE's}
\end{Highlighting}
\end{Shaded}

\begin{verbatim}
##       mean          sd    
##   3.26428571   0.31246442 
##  (0.06818533) (0.04821431)
\end{verbatim}

\begin{Shaded}
\begin{Highlighting}[]
\NormalTok{x <-}\StringTok{ }\DecValTok{0}\OperatorTok{:}\DecValTok{7} \CommentTok{# range of x values across where you want distribution}
\NormalTok{p_density <-}\StringTok{ }\KeywordTok{dnorm}\NormalTok{(}\DataTypeTok{x=}\NormalTok{x,}
                   \DataTypeTok{mean=}\NormalTok{z}\OperatorTok{$}\NormalTok{estimate[}\StringTok{"mean"}\NormalTok{],}
                   \DataTypeTok{sd=}\NormalTok{z}\OperatorTok{$}\NormalTok{estimate[}\StringTok{"sd"}\NormalTok{]) }\CommentTok{# use mean and sd from actual data}
\KeywordTok{qplot}\NormalTok{(x,p_density,}\DataTypeTok{geom=}\StringTok{"line"}\NormalTok{) }\OperatorTok{+}\StringTok{ }\KeywordTok{annotate}\NormalTok{(}\DataTypeTok{geom=}\StringTok{"point"}\NormalTok{,}\DataTypeTok{x=}\NormalTok{pro_data,}\DataTypeTok{y=}\FloatTok{0.0000001}\NormalTok{,}\DataTypeTok{color=}\StringTok{"royalblue2"}\NormalTok{) }\OperatorTok{+}\StringTok{ }\KeywordTok{labs}\NormalTok{(}\DataTypeTok{title=}\StringTok{"Percent protein data against a normal distribution density curve"}\NormalTok{,}
       \DataTypeTok{x=}\StringTok{"BTM Percent protein of 21 herds"}\NormalTok{,}
       \DataTypeTok{y=}\StringTok{"Probability density of normal dist."}\NormalTok{) }\CommentTok{# see data points (x values) fall right in middle of probability mass curve}
\end{Highlighting}
\end{Shaded}

\includegraphics{probability_dist_BTM_40herd_edited_files/figure-latex/unnamed-chunk-14-1.pdf}

\begin{Shaded}
\begin{Highlighting}[]
\CommentTok{# do the same for the gamma distribution}
\NormalTok{z <-}\KeywordTok{fitdistr}\NormalTok{(pro_data,}\StringTok{"gamma"}\NormalTok{)}
\KeywordTok{print}\NormalTok{(z)}
\end{Highlighting}
\end{Shaded}

\begin{verbatim}
##      shape       rate   
##   110.85796    33.96085 
##  ( 34.15802) ( 10.48780)
\end{verbatim}

\begin{Shaded}
\begin{Highlighting}[]
\NormalTok{p_density <-}\StringTok{ }\KeywordTok{dgamma}\NormalTok{(}\DataTypeTok{x=}\NormalTok{x,}
                   \DataTypeTok{shape=}\NormalTok{z}\OperatorTok{$}\NormalTok{estimate[}\StringTok{"shape"}\NormalTok{],}
                   \DataTypeTok{rate=}\NormalTok{z}\OperatorTok{$}\NormalTok{estimate[}\StringTok{"rate"}\NormalTok{])}
\KeywordTok{qplot}\NormalTok{(x,p_density,}\DataTypeTok{geom=}\StringTok{"line"}\NormalTok{) }\OperatorTok{+}\StringTok{ }\KeywordTok{annotate}\NormalTok{(}\DataTypeTok{geom=}\StringTok{"point"}\NormalTok{,}\DataTypeTok{x=}\NormalTok{pro_data,}\DataTypeTok{y=}\FloatTok{0.0000001}\NormalTok{, }\DataTypeTok{color=}\StringTok{"royalblue2"}\NormalTok{) }\OperatorTok{+}\StringTok{ }\KeywordTok{labs}\NormalTok{(}\DataTypeTok{title=}\StringTok{"Percent protein data against a gamma distribution density curve"}\NormalTok{,}
       \DataTypeTok{x=}\StringTok{"BTM Percent protein of 21 herds"}\NormalTok{,}
       \DataTypeTok{y=}\StringTok{"Probability density of gamma dist."}\NormalTok{) }\CommentTok{#bound at 0}
\end{Highlighting}
\end{Shaded}

\includegraphics{probability_dist_BTM_40herd_edited_files/figure-latex/unnamed-chunk-14-2.pdf}

\begin{Shaded}
\begin{Highlighting}[]
\KeywordTok{ggdensity}\NormalTok{(btm}\OperatorTok{$}\NormalTok{perc_Protein, }
          \DataTypeTok{main =} \StringTok{"Density plot of BTM % Protein"}\NormalTok{,}
          \DataTypeTok{xlab =} \StringTok{"% Protein of 21 herds"}\NormalTok{)}
\end{Highlighting}
\end{Shaded}

\includegraphics{probability_dist_BTM_40herd_edited_files/figure-latex/unnamed-chunk-14-3.pdf}

\begin{Shaded}
\begin{Highlighting}[]
\KeywordTok{ggqqplot}\NormalTok{(btm}\OperatorTok{$}\NormalTok{perc_Protein)}
\end{Highlighting}
\end{Shaded}

\includegraphics{probability_dist_BTM_40herd_edited_files/figure-latex/unnamed-chunk-14-4.pdf}

\subsection{\texorpdfstring{\textbf{\emph{Percent
lactose}}}{Percent lactose}}\label{percent-lactose}

\begin{Shaded}
\begin{Highlighting}[]
\KeywordTok{summary}\NormalTok{(btm}\OperatorTok{$}\NormalTok{perc_Lactose)}
\end{Highlighting}
\end{Shaded}

\begin{verbatim}
##    Min. 1st Qu.  Median    Mean 3rd Qu.    Max. 
##   4.530   4.570   4.620   4.614   4.650   4.710
\end{verbatim}

\begin{Shaded}
\begin{Highlighting}[]
\NormalTok{lac_data<-btm}\OperatorTok{$}\NormalTok{perc_Lactose}
\NormalTok{z <-}\KeywordTok{fitdistr}\NormalTok{(lac_data,}\StringTok{"normal"}\NormalTok{)}
\KeywordTok{print}\NormalTok{(z) }\CommentTok{# mean and SD, the MLE's}
\end{Highlighting}
\end{Shaded}

\begin{verbatim}
##       mean           sd     
##   4.614285714   0.049528388 
##  (0.010807980) (0.007642396)
\end{verbatim}

\begin{Shaded}
\begin{Highlighting}[]
\NormalTok{x <-}\StringTok{ }\DecValTok{4}\OperatorTok{:}\DecValTok{6} \CommentTok{# range of x values across where you want distribution}
\NormalTok{p_density <-}\StringTok{ }\KeywordTok{dnorm}\NormalTok{(}\DataTypeTok{x=}\NormalTok{x,}
                   \DataTypeTok{mean=}\NormalTok{z}\OperatorTok{$}\NormalTok{estimate[}\StringTok{"mean"}\NormalTok{],}
                   \DataTypeTok{sd=}\NormalTok{z}\OperatorTok{$}\NormalTok{estimate[}\StringTok{"sd"}\NormalTok{]) }\CommentTok{# use mean and sd from actual data}
\KeywordTok{qplot}\NormalTok{(x,p_density,}\DataTypeTok{geom=}\StringTok{"line"}\NormalTok{) }\OperatorTok{+}\StringTok{ }\KeywordTok{annotate}\NormalTok{(}\DataTypeTok{geom=}\StringTok{"point"}\NormalTok{,}\DataTypeTok{x=}\NormalTok{lac_data,}\DataTypeTok{y=}\FloatTok{0.00000000000001}\NormalTok{,}\DataTypeTok{color=}\StringTok{"royalblue3"}\NormalTok{) }\OperatorTok{+}\StringTok{ }\KeywordTok{labs}\NormalTok{(}\DataTypeTok{title=}\StringTok{"Percent lactose data against a normal distribution density curve"}\NormalTok{,}
       \DataTypeTok{x=}\StringTok{"BTM Percent lactose of 21 herds"}\NormalTok{,}
       \DataTypeTok{y=}\StringTok{"Probability density of normal dist."}\NormalTok{) }\CommentTok{# see data points (x values) fall right in middle of probability mass curve}
\end{Highlighting}
\end{Shaded}

\includegraphics{probability_dist_BTM_40herd_edited_files/figure-latex/unnamed-chunk-15-1.pdf}

\begin{Shaded}
\begin{Highlighting}[]
\CommentTok{# do the same for the gamma distribution}
\NormalTok{z <-}\KeywordTok{fitdistr}\NormalTok{(lac_data,}\StringTok{"gamma"}\NormalTok{)}
\KeywordTok{print}\NormalTok{(z)}
\end{Highlighting}
\end{Shaded}

\begin{verbatim}
##      shape       rate   
##   8266.3022   1791.4587 
##  (2508.5328) ( 543.6614)
\end{verbatim}

\begin{Shaded}
\begin{Highlighting}[]
\NormalTok{p_density <-}\StringTok{ }\KeywordTok{dgamma}\NormalTok{(}\DataTypeTok{x=}\NormalTok{x,}
                   \DataTypeTok{shape=}\NormalTok{z}\OperatorTok{$}\NormalTok{estimate[}\StringTok{"shape"}\NormalTok{],}
                   \DataTypeTok{rate=}\NormalTok{z}\OperatorTok{$}\NormalTok{estimate[}\StringTok{"rate"}\NormalTok{])}
\KeywordTok{qplot}\NormalTok{(x,p_density,}\DataTypeTok{geom=}\StringTok{"line"}\NormalTok{) }\OperatorTok{+}\StringTok{ }\KeywordTok{annotate}\NormalTok{(}\DataTypeTok{geom=}\StringTok{"point"}\NormalTok{,}\DataTypeTok{x=}\NormalTok{lac_data,}\DataTypeTok{y=}\FloatTok{0.00000000000001}\NormalTok{, }\DataTypeTok{color=}\StringTok{"royalblue3"}\NormalTok{) }\OperatorTok{+}\StringTok{ }\KeywordTok{labs}\NormalTok{(}\DataTypeTok{title=}\StringTok{"Percent lactose data against a gamma distribution density curve"}\NormalTok{,}
       \DataTypeTok{x=}\StringTok{"BTM Percent lactose of 21 herds"}\NormalTok{,}
       \DataTypeTok{y=}\StringTok{"Probability density of gamma dist."}\NormalTok{) }\CommentTok{#bound at 0}
\end{Highlighting}
\end{Shaded}

\includegraphics{probability_dist_BTM_40herd_edited_files/figure-latex/unnamed-chunk-15-2.pdf}

\begin{Shaded}
\begin{Highlighting}[]
\KeywordTok{ggdensity}\NormalTok{(btm}\OperatorTok{$}\NormalTok{perc_Lactose, }
          \DataTypeTok{main =} \StringTok{"Density plot of BTM % Lactose"}\NormalTok{,}
          \DataTypeTok{xlab =} \StringTok{"% Lactose of 21 herds"}\NormalTok{)}
\end{Highlighting}
\end{Shaded}

\includegraphics{probability_dist_BTM_40herd_edited_files/figure-latex/unnamed-chunk-15-3.pdf}

\begin{Shaded}
\begin{Highlighting}[]
\KeywordTok{ggqqplot}\NormalTok{(btm}\OperatorTok{$}\NormalTok{perc_Lactose)}
\end{Highlighting}
\end{Shaded}

\includegraphics{probability_dist_BTM_40herd_edited_files/figure-latex/unnamed-chunk-15-4.pdf}

\subsection{\texorpdfstring{\textbf{\emph{Percent other
solids}}}{Percent other solids}}\label{percent-other-solids}

\begin{Shaded}
\begin{Highlighting}[]
\KeywordTok{summary}\NormalTok{(btm}\OperatorTok{$}\NormalTok{perc_Other.Solids)}
\end{Highlighting}
\end{Shaded}

\begin{verbatim}
##    Min. 1st Qu.  Median    Mean 3rd Qu.    Max. 
##    5.62    5.68    5.71    5.72    5.74    5.86
\end{verbatim}

\begin{Shaded}
\begin{Highlighting}[]
\NormalTok{solids_data<-btm}\OperatorTok{$}\NormalTok{perc_Other.Solids}
\NormalTok{z <-}\KeywordTok{fitdistr}\NormalTok{(solids_data,}\StringTok{"normal"}\NormalTok{)}
\KeywordTok{print}\NormalTok{(z) }\CommentTok{# mean and SD, the MLE's}
\end{Highlighting}
\end{Shaded}

\begin{verbatim}
##       mean           sd     
##   5.719523810   0.059799362 
##  (0.013049291) (0.009227242)
\end{verbatim}

\begin{Shaded}
\begin{Highlighting}[]
\NormalTok{x <-}\StringTok{ }\DecValTok{4}\OperatorTok{:}\DecValTok{8} \CommentTok{# range of x values across where you want distribution}
\NormalTok{p_density <-}\StringTok{ }\KeywordTok{dnorm}\NormalTok{(}\DataTypeTok{x=}\NormalTok{x,}
                   \DataTypeTok{mean=}\NormalTok{z}\OperatorTok{$}\NormalTok{estimate[}\StringTok{"mean"}\NormalTok{],}
                   \DataTypeTok{sd=}\NormalTok{z}\OperatorTok{$}\NormalTok{estimate[}\StringTok{"sd"}\NormalTok{]) }\CommentTok{# use mean and sd from actual data}
\KeywordTok{qplot}\NormalTok{(x,p_density,}\DataTypeTok{geom=}\StringTok{"line"}\NormalTok{) }\OperatorTok{+}\StringTok{ }\KeywordTok{annotate}\NormalTok{(}\DataTypeTok{geom=}\StringTok{"point"}\NormalTok{,}\DataTypeTok{x=}\NormalTok{solids_data,}\DataTypeTok{y=}\FloatTok{0.00000000000001}\NormalTok{,}\DataTypeTok{color=}\StringTok{"royalblue4"}\NormalTok{) }\OperatorTok{+}\StringTok{ }\KeywordTok{labs}\NormalTok{(}\DataTypeTok{title=}\StringTok{"Percent other solids data against a normal distribution density curve"}\NormalTok{,}
       \DataTypeTok{x=}\StringTok{"BTM Percent other solids of 21 herds"}\NormalTok{,}
       \DataTypeTok{y=}\StringTok{"Probability density of normal dist."}\NormalTok{) }\CommentTok{# see data points (x values) fall right in middle of probability mass curve}
\end{Highlighting}
\end{Shaded}

\includegraphics{probability_dist_BTM_40herd_edited_files/figure-latex/unnamed-chunk-16-1.pdf}

\begin{Shaded}
\begin{Highlighting}[]
\CommentTok{# do the same for the gamma distribution}
\NormalTok{z <-}\KeywordTok{fitdistr}\NormalTok{(solids_data,}\StringTok{"gamma"}\NormalTok{)}
\KeywordTok{print}\NormalTok{(z)}
\end{Highlighting}
\end{Shaded}

\begin{verbatim}
##      shape       rate   
##   8712.3907   1523.2720 
##  (2684.0936) ( 469.2993)
\end{verbatim}

\begin{Shaded}
\begin{Highlighting}[]
\NormalTok{p_density <-}\StringTok{ }\KeywordTok{dgamma}\NormalTok{(}\DataTypeTok{x=}\NormalTok{x,}
                   \DataTypeTok{shape=}\NormalTok{z}\OperatorTok{$}\NormalTok{estimate[}\StringTok{"shape"}\NormalTok{],}
                   \DataTypeTok{rate=}\NormalTok{z}\OperatorTok{$}\NormalTok{estimate[}\StringTok{"rate"}\NormalTok{])}
\KeywordTok{qplot}\NormalTok{(x,p_density,}\DataTypeTok{geom=}\StringTok{"line"}\NormalTok{) }\OperatorTok{+}\StringTok{ }\KeywordTok{annotate}\NormalTok{(}\DataTypeTok{geom=}\StringTok{"point"}\NormalTok{,}\DataTypeTok{x=}\NormalTok{solids_data,}\DataTypeTok{y=}\FloatTok{0.00000000000001}\NormalTok{, }\DataTypeTok{color=}\StringTok{"royalblue4"}\NormalTok{) }\OperatorTok{+}\StringTok{ }\KeywordTok{labs}\NormalTok{(}\DataTypeTok{title=}\StringTok{"Percent other solids data against a gamma distribution density curve"}\NormalTok{,}
       \DataTypeTok{x=}\StringTok{"BTM Percent other solids of 21 herds"}\NormalTok{,}
       \DataTypeTok{y=}\StringTok{"Probability density of gamma dist."}\NormalTok{) }\CommentTok{#bound at 0}
\end{Highlighting}
\end{Shaded}

\includegraphics{probability_dist_BTM_40herd_edited_files/figure-latex/unnamed-chunk-16-2.pdf}

\begin{Shaded}
\begin{Highlighting}[]
\KeywordTok{ggdensity}\NormalTok{(btm}\OperatorTok{$}\NormalTok{perc_Other.Solids, }
          \DataTypeTok{main =} \StringTok{"Density plot of BTM % Other solids"}\NormalTok{,}
          \DataTypeTok{xlab =} \StringTok{"% Other solids of 21 herds"}\NormalTok{)}
\end{Highlighting}
\end{Shaded}

\includegraphics{probability_dist_BTM_40herd_edited_files/figure-latex/unnamed-chunk-16-3.pdf}

\begin{Shaded}
\begin{Highlighting}[]
\KeywordTok{ggqqplot}\NormalTok{(btm}\OperatorTok{$}\NormalTok{perc_Other.Solids)}
\end{Highlighting}
\end{Shaded}

\includegraphics{probability_dist_BTM_40herd_edited_files/figure-latex/unnamed-chunk-16-4.pdf}

\subsection{\texorpdfstring{\textbf{\emph{MUN}}}{MUN}}\label{mun}

\begin{Shaded}
\begin{Highlighting}[]
\KeywordTok{summary}\NormalTok{(btm}\OperatorTok{$}\NormalTok{MUN)}
\end{Highlighting}
\end{Shaded}

\begin{verbatim}
##    Min. 1st Qu.  Median    Mean 3rd Qu.    Max. 
##    6.51    9.69   10.94   11.57   13.26   17.05
\end{verbatim}

\begin{Shaded}
\begin{Highlighting}[]
\NormalTok{mun_data<-btm}\OperatorTok{$}\NormalTok{MUN}
\NormalTok{z <-}\KeywordTok{fitdistr}\NormalTok{(mun_data,}\StringTok{"normal"}\NormalTok{)}
\KeywordTok{print}\NormalTok{(z) }\CommentTok{# mean and SD, the MLE's}
\end{Highlighting}
\end{Shaded}

\begin{verbatim}
##       mean          sd    
##   11.5676190    2.4762354 
##  ( 0.5403589) ( 0.3820914)
\end{verbatim}

\begin{Shaded}
\begin{Highlighting}[]
\NormalTok{x <-}\StringTok{ }\DecValTok{0}\OperatorTok{:}\DecValTok{25} \CommentTok{# range of x values across where you want distribution}
\NormalTok{p_density <-}\StringTok{ }\KeywordTok{dnorm}\NormalTok{(}\DataTypeTok{x=}\NormalTok{x,}
                   \DataTypeTok{mean=}\NormalTok{z}\OperatorTok{$}\NormalTok{estimate[}\StringTok{"mean"}\NormalTok{],}
                   \DataTypeTok{sd=}\NormalTok{z}\OperatorTok{$}\NormalTok{estimate[}\StringTok{"sd"}\NormalTok{]) }\CommentTok{# use mean and sd from actual data}
\KeywordTok{qplot}\NormalTok{(x,p_density,}\DataTypeTok{geom=}\StringTok{"line"}\NormalTok{) }\OperatorTok{+}\StringTok{ }\KeywordTok{annotate}\NormalTok{(}\DataTypeTok{geom=}\StringTok{"point"}\NormalTok{,}\DataTypeTok{x=}\NormalTok{mun_data,}\DataTypeTok{y=}\FloatTok{0.001}\NormalTok{,}\DataTypeTok{color=}\StringTok{"firebrick2"}\NormalTok{) }\OperatorTok{+}\StringTok{ }\KeywordTok{labs}\NormalTok{(}\DataTypeTok{title=}\StringTok{"MUN data against a normal distribution density curve"}\NormalTok{,}
       \DataTypeTok{x=}\StringTok{"BTM MUN of 21 herds"}\NormalTok{,}
       \DataTypeTok{y=}\StringTok{"Probability density of normal dist."}\NormalTok{) }\CommentTok{# see data points (x values) fall right in middle of probability mass curve}
\end{Highlighting}
\end{Shaded}

\includegraphics{probability_dist_BTM_40herd_edited_files/figure-latex/unnamed-chunk-17-1.pdf}

\begin{Shaded}
\begin{Highlighting}[]
\CommentTok{# do the same for the gamma distribution}
\NormalTok{z <-}\KeywordTok{fitdistr}\NormalTok{(mun_data,}\StringTok{"gamma"}\NormalTok{)}
\KeywordTok{print}\NormalTok{(z)}
\end{Highlighting}
\end{Shaded}

\begin{verbatim}
##      shape       rate   
##   21.613294    1.868431 
##  ( 6.619068) ( 0.578888)
\end{verbatim}

\begin{Shaded}
\begin{Highlighting}[]
\NormalTok{p_density <-}\StringTok{ }\KeywordTok{dgamma}\NormalTok{(}\DataTypeTok{x=}\NormalTok{x,}
                   \DataTypeTok{shape=}\NormalTok{z}\OperatorTok{$}\NormalTok{estimate[}\StringTok{"shape"}\NormalTok{],}
                   \DataTypeTok{rate=}\NormalTok{z}\OperatorTok{$}\NormalTok{estimate[}\StringTok{"rate"}\NormalTok{])}
\KeywordTok{qplot}\NormalTok{(x,p_density,}\DataTypeTok{geom=}\StringTok{"line"}\NormalTok{) }\OperatorTok{+}\StringTok{ }\KeywordTok{annotate}\NormalTok{(}\DataTypeTok{geom=}\StringTok{"point"}\NormalTok{,}\DataTypeTok{x=}\NormalTok{mun_data,}\DataTypeTok{y=}\FloatTok{0.001}\NormalTok{, }\DataTypeTok{color=}\StringTok{"firebrick2"}\NormalTok{) }\OperatorTok{+}\StringTok{ }\KeywordTok{labs}\NormalTok{(}\DataTypeTok{title=}\StringTok{"MUN data against a gamma distribution density curve"}\NormalTok{,}
       \DataTypeTok{x=}\StringTok{"BTM MUN of 21 herds"}\NormalTok{,}
       \DataTypeTok{y=}\StringTok{"Probability density of gamma dist."}\NormalTok{)}
\end{Highlighting}
\end{Shaded}

\includegraphics{probability_dist_BTM_40herd_edited_files/figure-latex/unnamed-chunk-17-2.pdf}

\begin{Shaded}
\begin{Highlighting}[]
\KeywordTok{ggdensity}\NormalTok{(btm}\OperatorTok{$}\NormalTok{MUN, }
          \DataTypeTok{main =} \StringTok{"Density plot of BTM MUN"}\NormalTok{,}
          \DataTypeTok{xlab =} \StringTok{"MUN of 21 herds"}\NormalTok{)}
\end{Highlighting}
\end{Shaded}

\includegraphics{probability_dist_BTM_40herd_edited_files/figure-latex/unnamed-chunk-17-3.pdf}

\begin{Shaded}
\begin{Highlighting}[]
\KeywordTok{ggqqplot}\NormalTok{(btm}\OperatorTok{$}\NormalTok{MUN)}
\end{Highlighting}
\end{Shaded}

\includegraphics{probability_dist_BTM_40herd_edited_files/figure-latex/unnamed-chunk-17-4.pdf}

\subsection{\texorpdfstring{\textbf{\emph{Minnesota
coliforms}}}{Minnesota coliforms}}\label{minnesota-coliforms}

\begin{Shaded}
\begin{Highlighting}[]
\KeywordTok{summary}\NormalTok{(btm}\OperatorTok{$}\NormalTok{Coliforms)}
\end{Highlighting}
\end{Shaded}

\begin{verbatim}
##    Min. 1st Qu.  Median    Mean 3rd Qu.    Max. 
##    0.00    0.00    0.00    1.19    0.00    5.00
\end{verbatim}

\begin{Shaded}
\begin{Highlighting}[]
\NormalTok{mcoli_data<-btm}\OperatorTok{$}\NormalTok{Coliforms}
\NormalTok{z <-}\KeywordTok{fitdistr}\NormalTok{(mcoli_data,}\StringTok{"normal"}\NormalTok{)}
\KeywordTok{print}\NormalTok{(z) }\CommentTok{# mean and SD, the MLE's}
\end{Highlighting}
\end{Shaded}

\begin{verbatim}
##      mean         sd    
##   1.1904762   2.1295885 
##  (0.4647143) (0.3286026)
\end{verbatim}

\begin{Shaded}
\begin{Highlighting}[]
\NormalTok{x <-}\StringTok{ }\DecValTok{0}\OperatorTok{:}\DecValTok{10} \CommentTok{# range of x values across where you want distribution}
\NormalTok{p_density <-}\StringTok{ }\KeywordTok{dnorm}\NormalTok{(}\DataTypeTok{x=}\NormalTok{x,}
                   \DataTypeTok{mean=}\NormalTok{z}\OperatorTok{$}\NormalTok{estimate[}\StringTok{"mean"}\NormalTok{],}
                   \DataTypeTok{sd=}\NormalTok{z}\OperatorTok{$}\NormalTok{estimate[}\StringTok{"sd"}\NormalTok{]) }\CommentTok{# use mean and sd from actual data}
\KeywordTok{qplot}\NormalTok{(x,p_density,}\DataTypeTok{geom=}\StringTok{"line"}\NormalTok{) }\OperatorTok{+}\StringTok{ }\KeywordTok{annotate}\NormalTok{(}\DataTypeTok{geom=}\StringTok{"point"}\NormalTok{, }\DataTypeTok{x=}\NormalTok{mcoli_data,}\DataTypeTok{y=}\FloatTok{0.00000000000001}\NormalTok{,}\DataTypeTok{color=}\StringTok{"firebrick"}\NormalTok{) }\OperatorTok{+}\StringTok{ }\KeywordTok{labs}\NormalTok{(}\DataTypeTok{title=}\StringTok{"Coliforms (Minn.) data against a normal distribution density curve"}\NormalTok{,}
       \DataTypeTok{x=}\StringTok{"BTM Coliforms (Minn) of 21 herds"}\NormalTok{,}
       \DataTypeTok{y=}\StringTok{"Probability density of normal dist."}\NormalTok{) }\CommentTok{# see data points (x values) fall right in middle of probability mass curve}
\end{Highlighting}
\end{Shaded}

\includegraphics{probability_dist_BTM_40herd_edited_files/figure-latex/unnamed-chunk-18-1.pdf}

\begin{Shaded}
\begin{Highlighting}[]
\NormalTok{mcoli_data_no0<-mcoli_data[mcoli_data}\OperatorTok{>}\DecValTok{0}\NormalTok{]}

\CommentTok{# # do the same for the gamma distribution}
\CommentTok{# z <-fitdistr(mcoli_data_no0,"gamma")}
\CommentTok{# print(z)}
\CommentTok{# p_density <- dgamma(x=x,}
\CommentTok{#                    shape=z$estimate["shape"],}
\CommentTok{#                    rate=z$estimate["rate"])}
\CommentTok{# qplot(x,p_density,geom="line") + annotate(geom="point",x=mcoli_data_no0,y=0.00000000000001, color="firebrick") + labs(title="Coliforms (Minn.) data against a gamma distribution density curve",}
\CommentTok{#        x="BTM Coliforms (Minn.)  of 21 herds",}
\CommentTok{#        y="Probability density of gamma dist.") #bound at 0}


\KeywordTok{ggdensity}\NormalTok{(btm}\OperatorTok{$}\NormalTok{Coliforms, }
          \DataTypeTok{main =} \StringTok{"Density plot of BTM Coliforms (Minn)"}\NormalTok{,}
          \DataTypeTok{xlab =} \StringTok{"Coliform counts of 21 herds"}\NormalTok{)}
\end{Highlighting}
\end{Shaded}

\includegraphics{probability_dist_BTM_40herd_edited_files/figure-latex/unnamed-chunk-18-2.pdf}

\begin{Shaded}
\begin{Highlighting}[]
\KeywordTok{ggqqplot}\NormalTok{(btm}\OperatorTok{$}\NormalTok{Coliforms)}
\end{Highlighting}
\end{Shaded}

\includegraphics{probability_dist_BTM_40herd_edited_files/figure-latex/unnamed-chunk-18-3.pdf}

\subsection{\texorpdfstring{\textbf{\emph{Minnesota non-ag
Streps}}}{Minnesota non-ag Streps}}\label{minnesota-non-ag-streps}

\begin{Shaded}
\begin{Highlighting}[]
\KeywordTok{summary}\NormalTok{(btm}\OperatorTok{$}\NormalTok{Non_ag_Strep)}
\end{Highlighting}
\end{Shaded}

\begin{verbatim}
##    Min. 1st Qu.  Median    Mean 3rd Qu.    Max. 
##    10.0    35.0    45.0   156.4   205.0  1250.0
\end{verbatim}

\begin{Shaded}
\begin{Highlighting}[]
\NormalTok{strep_data<-btm}\OperatorTok{$}\NormalTok{Non_ag_Strep}
\NormalTok{z <-}\KeywordTok{fitdistr}\NormalTok{(strep_data,}\StringTok{"normal"}\NormalTok{)}
\KeywordTok{print}\NormalTok{(z) }\CommentTok{# mean and SD, the MLE's}
\end{Highlighting}
\end{Shaded}

\begin{verbatim}
##      mean         sd    
##   156.42857   261.94394 
##  ( 57.16085) ( 40.41883)
\end{verbatim}

\begin{Shaded}
\begin{Highlighting}[]
\NormalTok{x <-}\StringTok{ }\DecValTok{0}\OperatorTok{:}\DecValTok{2000} \CommentTok{# range of x values across where you want distribution}
\NormalTok{p_density <-}\StringTok{ }\KeywordTok{dnorm}\NormalTok{(}\DataTypeTok{x=}\NormalTok{x,}
                   \DataTypeTok{mean=}\NormalTok{z}\OperatorTok{$}\NormalTok{estimate[}\StringTok{"mean"}\NormalTok{],}
                   \DataTypeTok{sd=}\NormalTok{z}\OperatorTok{$}\NormalTok{estimate[}\StringTok{"sd"}\NormalTok{]) }\CommentTok{# use mean and sd from actual data}
\KeywordTok{qplot}\NormalTok{(x,p_density,}\DataTypeTok{geom=}\StringTok{"line"}\NormalTok{) }\OperatorTok{+}\StringTok{ }\KeywordTok{annotate}\NormalTok{(}\DataTypeTok{geom=}\StringTok{"point"}\NormalTok{,}\DataTypeTok{x=}\NormalTok{strep_data,}\DataTypeTok{y=}\FloatTok{0.0000001}\NormalTok{,}\DataTypeTok{color=}\StringTok{"firebrick2"}\NormalTok{) }\OperatorTok{+}\StringTok{ }\KeywordTok{labs}\NormalTok{(}\DataTypeTok{title=}\StringTok{"Non-ag Streps (Minn.) data against a normal distribution density curve"}\NormalTok{,}
       \DataTypeTok{x=}\StringTok{"BTM non-ag Streps (Minn) of 21 herds"}\NormalTok{,}
       \DataTypeTok{y=}\StringTok{"Probability density of normal dist."}\NormalTok{) }\CommentTok{# see data points (x values) fall right in middle of probability mass curve}
\end{Highlighting}
\end{Shaded}

\includegraphics{probability_dist_BTM_40herd_edited_files/figure-latex/unnamed-chunk-19-1.pdf}

\begin{Shaded}
\begin{Highlighting}[]
\CommentTok{# do the same for the gamma distribution}
\NormalTok{z <-}\KeywordTok{fitdistr}\NormalTok{(strep_data,}\StringTok{"gamma"}\NormalTok{)}
\KeywordTok{print}\NormalTok{(z)}
\end{Highlighting}
\end{Shaded}

\begin{verbatim}
##       shape         rate    
##   0.784269959   0.005013601 
##  (0.205274781) (0.001753532)
\end{verbatim}

\begin{Shaded}
\begin{Highlighting}[]
\NormalTok{p_density <-}\StringTok{ }\KeywordTok{dgamma}\NormalTok{(}\DataTypeTok{x=}\NormalTok{x,}
                   \DataTypeTok{shape=}\NormalTok{z}\OperatorTok{$}\NormalTok{estimate[}\StringTok{"shape"}\NormalTok{],}
                   \DataTypeTok{rate=}\NormalTok{z}\OperatorTok{$}\NormalTok{estimate[}\StringTok{"rate"}\NormalTok{])}
\KeywordTok{qplot}\NormalTok{(x,p_density,}\DataTypeTok{geom=}\StringTok{"line"}\NormalTok{) }\OperatorTok{+}\StringTok{ }\KeywordTok{annotate}\NormalTok{(}\DataTypeTok{geom=}\StringTok{"point"}\NormalTok{,}\DataTypeTok{x=}\NormalTok{strep_data,}\DataTypeTok{y=}\FloatTok{0.00000000000001}\NormalTok{, }\DataTypeTok{color=}\StringTok{"firebrick2"}\NormalTok{) }\OperatorTok{+}\StringTok{ }\KeywordTok{labs}\NormalTok{(}\DataTypeTok{title=}\StringTok{"Non-ag Streps (Minn.) data against a gamma distribution density curve"}\NormalTok{,}
       \DataTypeTok{x=}\StringTok{"BTM non-ag Streps (Minn.)  of 21 herds"}\NormalTok{,}
       \DataTypeTok{y=}\StringTok{"Probability density of gamma dist."}\NormalTok{) }\CommentTok{#bound at 0}
\end{Highlighting}
\end{Shaded}

\includegraphics{probability_dist_BTM_40herd_edited_files/figure-latex/unnamed-chunk-19-2.pdf}

\begin{Shaded}
\begin{Highlighting}[]
\KeywordTok{ggdensity}\NormalTok{(btm}\OperatorTok{$}\NormalTok{Non_ag_Strep, }
          \DataTypeTok{main =} \StringTok{"Density plot of BTM Non-ag Streps (Minn)"}\NormalTok{,}
          \DataTypeTok{xlab =} \StringTok{"Non-ag Strep counts of 21 herds"}\NormalTok{)}
\end{Highlighting}
\end{Shaded}

\includegraphics{probability_dist_BTM_40herd_edited_files/figure-latex/unnamed-chunk-19-3.pdf}

\begin{Shaded}
\begin{Highlighting}[]
\KeywordTok{ggqqplot}\NormalTok{(btm}\OperatorTok{$}\NormalTok{Non_ag_Strep)}
\end{Highlighting}
\end{Shaded}

\includegraphics{probability_dist_BTM_40herd_edited_files/figure-latex/unnamed-chunk-19-4.pdf}

\begin{Shaded}
\begin{Highlighting}[]
\CommentTok{#---------------------------------------------------}

\CommentTok{# remove outlier}
\NormalTok{strep_data_trim<-strep_data[strep_data}\OperatorTok{<}\DecValTok{400}\NormalTok{]}
\NormalTok{z <-}\KeywordTok{fitdistr}\NormalTok{(strep_data_trim,}\StringTok{"normal"}\NormalTok{)}
\KeywordTok{print}\NormalTok{(z) }\CommentTok{# mean and SD, the MLE's}
\end{Highlighting}
\end{Shaded}

\begin{verbatim}
##      mean         sd    
##   101.75000    96.23247 
##  ( 21.51823) ( 15.21569)
\end{verbatim}

\begin{Shaded}
\begin{Highlighting}[]
\NormalTok{x <-}\StringTok{ }\DecValTok{0}\OperatorTok{:}\DecValTok{500} \CommentTok{# range of x values across where you want distribution}
\NormalTok{p_density <-}\StringTok{ }\KeywordTok{dnorm}\NormalTok{(}\DataTypeTok{x=}\NormalTok{x,}
                   \DataTypeTok{mean=}\NormalTok{z}\OperatorTok{$}\NormalTok{estimate[}\StringTok{"mean"}\NormalTok{],}
                   \DataTypeTok{sd=}\NormalTok{z}\OperatorTok{$}\NormalTok{estimate[}\StringTok{"sd"}\NormalTok{]) }\CommentTok{# use mean and sd from actual data}
\KeywordTok{qplot}\NormalTok{(x,p_density,}\DataTypeTok{geom=}\StringTok{"line"}\NormalTok{) }\OperatorTok{+}\StringTok{ }\KeywordTok{annotate}\NormalTok{(}\DataTypeTok{geom=}\StringTok{"point"}\NormalTok{,}\DataTypeTok{x=}\NormalTok{strep_data_trim,}\DataTypeTok{y=}\FloatTok{0.0000001}\NormalTok{,}\DataTypeTok{color=}\StringTok{"firebrick2"}\NormalTok{) }\OperatorTok{+}\StringTok{ }\KeywordTok{labs}\NormalTok{(}\DataTypeTok{title=}\StringTok{"Non-ag Streps (Minn.) data against a normal distribution density curve"}\NormalTok{,}
       \DataTypeTok{x=}\StringTok{"BTM non-ag Streps (Minn) of 21 herds"}\NormalTok{,}
       \DataTypeTok{y=}\StringTok{"Probability density of normal dist."}\NormalTok{) }\CommentTok{# see data points (x values) fall right in middle of probability mass curve}
\end{Highlighting}
\end{Shaded}

\includegraphics{probability_dist_BTM_40herd_edited_files/figure-latex/unnamed-chunk-19-5.pdf}

\begin{Shaded}
\begin{Highlighting}[]
\KeywordTok{ggqqplot}\NormalTok{(strep_data_trim)}
\end{Highlighting}
\end{Shaded}

\includegraphics{probability_dist_BTM_40herd_edited_files/figure-latex/unnamed-chunk-19-6.pdf}

\subsection{\texorpdfstring{\textbf{\emph{Minnesota non-ag Streps log10
transformation}}}{Minnesota non-ag Streps log10 transformation}}\label{minnesota-non-ag-streps-log10-transformation}

\begin{Shaded}
\begin{Highlighting}[]
\KeywordTok{summary}\NormalTok{(btm}\OperatorTok{$}\NormalTok{log10_strep)}
\end{Highlighting}
\end{Shaded}

\begin{verbatim}
##    Min. 1st Qu.  Median    Mean 3rd Qu.    Max. 
##   1.000   1.544   1.653   1.865   2.312   3.097
\end{verbatim}

\begin{Shaded}
\begin{Highlighting}[]
\NormalTok{streplog_data<-btm}\OperatorTok{$}\NormalTok{log10_strep}
\NormalTok{z <-}\KeywordTok{fitdistr}\NormalTok{(streplog_data,}\StringTok{"normal"}\NormalTok{)}
\KeywordTok{print}\NormalTok{(z) }\CommentTok{# mean and SD, the MLE's}
\end{Highlighting}
\end{Shaded}

\begin{verbatim}
##       mean          sd    
##   1.86480124   0.50115719 
##  (0.10936146) (0.07733023)
\end{verbatim}

\begin{Shaded}
\begin{Highlighting}[]
\NormalTok{x <-}\StringTok{ }\DecValTok{0}\OperatorTok{:}\DecValTok{10} \CommentTok{# range of x values across where you want distribution}
\NormalTok{p_density <-}\StringTok{ }\KeywordTok{dnorm}\NormalTok{(}\DataTypeTok{x=}\NormalTok{x,}
                   \DataTypeTok{mean=}\NormalTok{z}\OperatorTok{$}\NormalTok{estimate[}\StringTok{"mean"}\NormalTok{],}
                   \DataTypeTok{sd=}\NormalTok{z}\OperatorTok{$}\NormalTok{estimate[}\StringTok{"sd"}\NormalTok{]) }\CommentTok{# use mean and sd from actual data}
\KeywordTok{qplot}\NormalTok{(x,p_density,}\DataTypeTok{geom=}\StringTok{"line"}\NormalTok{) }\OperatorTok{+}\StringTok{ }\KeywordTok{annotate}\NormalTok{(}\DataTypeTok{geom=}\StringTok{"point"}\NormalTok{,}\DataTypeTok{x=}\NormalTok{streplog_data,}\DataTypeTok{y=}\FloatTok{0.0000001}\NormalTok{,}\DataTypeTok{color=}\StringTok{"firebrick2"}\NormalTok{) }\OperatorTok{+}\StringTok{ }\KeywordTok{labs}\NormalTok{(}\DataTypeTok{title=}\StringTok{"Non-ag Streps (Minn.) log10 data against a normal distribution density curve"}\NormalTok{,}
       \DataTypeTok{x=}\StringTok{"BTM non-ag Streps (Minn) log10 of 21 herds"}\NormalTok{,}
       \DataTypeTok{y=}\StringTok{"Probability density of normal dist."}\NormalTok{) }\CommentTok{# see data points (x values) fall right in middle of probability mass curve}
\end{Highlighting}
\end{Shaded}

\includegraphics{probability_dist_BTM_40herd_edited_files/figure-latex/unnamed-chunk-20-1.pdf}

\begin{Shaded}
\begin{Highlighting}[]
\CommentTok{# do the same for the gamma distribution}
\NormalTok{z <-}\KeywordTok{fitdistr}\NormalTok{(streplog_data,}\StringTok{"gamma"}\NormalTok{)}
\KeywordTok{print}\NormalTok{(z)}
\end{Highlighting}
\end{Shaded}

\begin{verbatim}
##      shape       rate   
##   14.250137    7.641638 
##  ( 4.347151) ( 2.372643)
\end{verbatim}

\begin{Shaded}
\begin{Highlighting}[]
\NormalTok{p_density <-}\StringTok{ }\KeywordTok{dgamma}\NormalTok{(}\DataTypeTok{x=}\NormalTok{x,}
                   \DataTypeTok{shape=}\NormalTok{z}\OperatorTok{$}\NormalTok{estimate[}\StringTok{"shape"}\NormalTok{],}
                   \DataTypeTok{rate=}\NormalTok{z}\OperatorTok{$}\NormalTok{estimate[}\StringTok{"rate"}\NormalTok{])}
\KeywordTok{qplot}\NormalTok{(x,p_density,}\DataTypeTok{geom=}\StringTok{"line"}\NormalTok{) }\OperatorTok{+}\StringTok{ }\KeywordTok{annotate}\NormalTok{(}\DataTypeTok{geom=}\StringTok{"point"}\NormalTok{,}\DataTypeTok{x=}\NormalTok{streplog_data,}\DataTypeTok{y=}\FloatTok{0.00000000000001}\NormalTok{, }\DataTypeTok{color=}\StringTok{"firebrick2"}\NormalTok{) }\OperatorTok{+}\StringTok{ }\KeywordTok{labs}\NormalTok{(}\DataTypeTok{title=}\StringTok{"Non-ag Streps (Minn.) log10 data against a gamma distribution density curve"}\NormalTok{,}
       \DataTypeTok{x=}\StringTok{"BTM non-ag Streps (Minn.) log10 of 21 herds"}\NormalTok{,}
       \DataTypeTok{y=}\StringTok{"Probability density of gamma dist."}\NormalTok{) }\CommentTok{#bound at 0}
\end{Highlighting}
\end{Shaded}

\includegraphics{probability_dist_BTM_40herd_edited_files/figure-latex/unnamed-chunk-20-2.pdf}

\begin{Shaded}
\begin{Highlighting}[]
\KeywordTok{ggdensity}\NormalTok{(btm}\OperatorTok{$}\NormalTok{log10_strep, }
          \DataTypeTok{main =} \StringTok{"Density plot of BTM Non-ag Streps (Minn) log10"}\NormalTok{,}
          \DataTypeTok{xlab =} \StringTok{"Non-ag Strep counts of 21 herds"}\NormalTok{)}
\end{Highlighting}
\end{Shaded}

\includegraphics{probability_dist_BTM_40herd_edited_files/figure-latex/unnamed-chunk-20-3.pdf}

\begin{Shaded}
\begin{Highlighting}[]
\KeywordTok{ggqqplot}\NormalTok{(btm}\OperatorTok{$}\NormalTok{log10_strep)}
\end{Highlighting}
\end{Shaded}

\includegraphics{probability_dist_BTM_40herd_edited_files/figure-latex/unnamed-chunk-20-4.pdf}

\subsection{\texorpdfstring{\textbf{\emph{Minnesota Staph.
aureus}}}{Minnesota Staph. aureus}}\label{minnesota-staph.-aureus}

\begin{Shaded}
\begin{Highlighting}[]
\KeywordTok{summary}\NormalTok{(btm}\OperatorTok{$}\NormalTok{Staph_aureus)}
\end{Highlighting}
\end{Shaded}

\begin{verbatim}
##    Min. 1st Qu.  Median    Mean 3rd Qu.    Max. 
##    0.00    0.00   30.00   43.57   55.00  320.00
\end{verbatim}

\begin{Shaded}
\begin{Highlighting}[]
\NormalTok{sa_data<-btm}\OperatorTok{$}\NormalTok{Staph_aureus}
\NormalTok{z <-}\KeywordTok{fitdistr}\NormalTok{(sa_data,}\StringTok{"normal"}\NormalTok{)}
\KeywordTok{print}\NormalTok{(z) }\CommentTok{# mean and SD, the MLE's}
\end{Highlighting}
\end{Shaded}

\begin{verbatim}
##      mean        sd   
##   43.57143   68.21086 
##  (14.88483) (10.52516)
\end{verbatim}

\begin{Shaded}
\begin{Highlighting}[]
\NormalTok{x <-}\StringTok{ }\DecValTok{0}\OperatorTok{:}\DecValTok{500} \CommentTok{# range of x values across where you want distribution}
\NormalTok{p_density <-}\StringTok{ }\KeywordTok{dnorm}\NormalTok{(}\DataTypeTok{x=}\NormalTok{x,}
                   \DataTypeTok{mean=}\NormalTok{z}\OperatorTok{$}\NormalTok{estimate[}\StringTok{"mean"}\NormalTok{],}
                   \DataTypeTok{sd=}\NormalTok{z}\OperatorTok{$}\NormalTok{estimate[}\StringTok{"sd"}\NormalTok{]) }\CommentTok{# use mean and sd from actual data}
\KeywordTok{qplot}\NormalTok{(x,p_density,}\DataTypeTok{geom=}\StringTok{"line"}\NormalTok{) }\OperatorTok{+}\StringTok{ }\KeywordTok{annotate}\NormalTok{(}\DataTypeTok{geom=}\StringTok{"point"}\NormalTok{,}\DataTypeTok{x=}\NormalTok{sa_data,}\DataTypeTok{y=}\FloatTok{0.0000001}\NormalTok{,}\DataTypeTok{color=}\StringTok{"firebrick3"}\NormalTok{) }\OperatorTok{+}\StringTok{ }\KeywordTok{labs}\NormalTok{(}\DataTypeTok{title=}\StringTok{"S. aureus (Minn.) data against a normal distribution density curve"}\NormalTok{,}
       \DataTypeTok{x=}\StringTok{"BTM S. aureus (Minn) of 21 herds"}\NormalTok{,}
       \DataTypeTok{y=}\StringTok{"Probability density of normal dist."}\NormalTok{) }\CommentTok{# see data points (x values) fall right in middle of probability mass curve}
\end{Highlighting}
\end{Shaded}

\includegraphics{probability_dist_BTM_40herd_edited_files/figure-latex/unnamed-chunk-21-1.pdf}

\begin{Shaded}
\begin{Highlighting}[]
\NormalTok{sa_data_no0<-sa_data[sa_data}\OperatorTok{>}\DecValTok{0}\NormalTok{]}
\CommentTok{# do the same for the gamma distribution, with 0's removed}
\NormalTok{z <-}\KeywordTok{fitdistr}\NormalTok{(sa_data_no0,}\StringTok{"gamma"}\NormalTok{)}
\KeywordTok{print}\NormalTok{(z)}
\end{Highlighting}
\end{Shaded}

\begin{verbatim}
##      shape         rate   
##   1.78316185   0.02533454 
##  (0.64364400) (0.01053816)
\end{verbatim}

\begin{Shaded}
\begin{Highlighting}[]
\NormalTok{p_density <-}\StringTok{ }\KeywordTok{dgamma}\NormalTok{(}\DataTypeTok{x=}\NormalTok{x,}
                   \DataTypeTok{shape=}\NormalTok{z}\OperatorTok{$}\NormalTok{estimate[}\StringTok{"shape"}\NormalTok{],}
                   \DataTypeTok{rate=}\NormalTok{z}\OperatorTok{$}\NormalTok{estimate[}\StringTok{"rate"}\NormalTok{])}
\KeywordTok{qplot}\NormalTok{(x,p_density,}\DataTypeTok{geom=}\StringTok{"line"}\NormalTok{) }\OperatorTok{+}\StringTok{ }\KeywordTok{annotate}\NormalTok{(}\DataTypeTok{geom=}\StringTok{"point"}\NormalTok{,}\DataTypeTok{x=}\NormalTok{sa_data_no0,}\DataTypeTok{y=}\FloatTok{0.0000001}\NormalTok{, }\DataTypeTok{color=}\StringTok{"firebrick3"}\NormalTok{) }\OperatorTok{+}\StringTok{ }\KeywordTok{labs}\NormalTok{(}\DataTypeTok{title=}\StringTok{"S. aureus (Minn.) data against a gamma distribution density curve"}\NormalTok{,}
       \DataTypeTok{x=}\StringTok{"BTM S. aureus (Minn.)  of 21 herds"}\NormalTok{,}
       \DataTypeTok{y=}\StringTok{"Probability density of gamma dist."}\NormalTok{) }\CommentTok{#bound at 0}
\end{Highlighting}
\end{Shaded}

\includegraphics{probability_dist_BTM_40herd_edited_files/figure-latex/unnamed-chunk-21-2.pdf}

\begin{Shaded}
\begin{Highlighting}[]
\KeywordTok{ggdensity}\NormalTok{(btm}\OperatorTok{$}\NormalTok{Staph_aureus, }
          \DataTypeTok{main =} \StringTok{"Density plot of BTM Staph. aureus (Minn)"}\NormalTok{,}
          \DataTypeTok{xlab =} \StringTok{"Staph. aureus counts of 21 herds"}\NormalTok{)}
\end{Highlighting}
\end{Shaded}

\includegraphics{probability_dist_BTM_40herd_edited_files/figure-latex/unnamed-chunk-21-3.pdf}

\begin{Shaded}
\begin{Highlighting}[]
\KeywordTok{ggqqplot}\NormalTok{(btm}\OperatorTok{$}\NormalTok{Staph_aureus)}
\end{Highlighting}
\end{Shaded}

\includegraphics{probability_dist_BTM_40herd_edited_files/figure-latex/unnamed-chunk-21-4.pdf}

\begin{Shaded}
\begin{Highlighting}[]
\CommentTok{#---------------------------------------------------}

\CommentTok{# remove outlier}
\NormalTok{sa_data_trim<-sa_data[sa_data}\OperatorTok{<}\DecValTok{200}\NormalTok{]}
\NormalTok{z <-}\KeywordTok{fitdistr}\NormalTok{(sa_data_trim,}\StringTok{"normal"}\NormalTok{)}
\KeywordTok{print}\NormalTok{(z) }\CommentTok{# mean and SD, the MLE's}
\end{Highlighting}
\end{Shaded}

\begin{verbatim}
##      mean         sd    
##   29.750000   29.558205 
##  ( 6.609416) ( 4.673563)
\end{verbatim}

\begin{Shaded}
\begin{Highlighting}[]
\NormalTok{x <-}\StringTok{ }\DecValTok{0}\OperatorTok{:}\DecValTok{200} \CommentTok{# range of x values across where you want distribution}
\NormalTok{p_density <-}\StringTok{ }\KeywordTok{dnorm}\NormalTok{(}\DataTypeTok{x=}\NormalTok{x,}
                   \DataTypeTok{mean=}\NormalTok{z}\OperatorTok{$}\NormalTok{estimate[}\StringTok{"mean"}\NormalTok{],}
                   \DataTypeTok{sd=}\NormalTok{z}\OperatorTok{$}\NormalTok{estimate[}\StringTok{"sd"}\NormalTok{]) }\CommentTok{# use mean and sd from actual data}
\KeywordTok{qplot}\NormalTok{(x,p_density,}\DataTypeTok{geom=}\StringTok{"line"}\NormalTok{) }\OperatorTok{+}\StringTok{ }\KeywordTok{annotate}\NormalTok{(}\DataTypeTok{geom=}\StringTok{"point"}\NormalTok{,}\DataTypeTok{x=}\NormalTok{sa_data_trim,}\DataTypeTok{y=}\FloatTok{0.0000001}\NormalTok{,}\DataTypeTok{color=}\StringTok{"firebrick3"}\NormalTok{) }\OperatorTok{+}\StringTok{ }\KeywordTok{labs}\NormalTok{(}\DataTypeTok{title=}\StringTok{"S. aureus (Minn.) data against a normal distribution density curve"}\NormalTok{,}
       \DataTypeTok{x=}\StringTok{"BTM S. aureus (Minn) of 21 herds"}\NormalTok{,}
       \DataTypeTok{y=}\StringTok{"Probability density of normal dist."}\NormalTok{) }\CommentTok{# see data points (x values) fall right in middle of probability mass curve}
\end{Highlighting}
\end{Shaded}

\includegraphics{probability_dist_BTM_40herd_edited_files/figure-latex/unnamed-chunk-21-5.pdf}

\begin{Shaded}
\begin{Highlighting}[]
\KeywordTok{ggqqplot}\NormalTok{(sa_data_trim)}
\end{Highlighting}
\end{Shaded}

\includegraphics{probability_dist_BTM_40herd_edited_files/figure-latex/unnamed-chunk-21-6.pdf}

\subsection{\texorpdfstring{\textbf{\emph{Minnesota Staph. aureus log10
transformation}}}{Minnesota Staph. aureus log10 transformation}}\label{minnesota-staph.-aureus-log10-transformation}

\begin{Shaded}
\begin{Highlighting}[]
\KeywordTok{summary}\NormalTok{(btm}\OperatorTok{$}\NormalTok{log10_aureus)}
\end{Highlighting}
\end{Shaded}

\begin{verbatim}
##    Min. 1st Qu.  Median    Mean 3rd Qu.    Max. 
##   0.000   0.000   1.477   1.061   1.740   2.505
\end{verbatim}

\begin{Shaded}
\begin{Highlighting}[]
\NormalTok{aureuslog_data<-btm}\OperatorTok{$}\NormalTok{log10_aureus}
\NormalTok{z <-}\KeywordTok{fitdistr}\NormalTok{(aureuslog_data,}\StringTok{"normal"}\NormalTok{)}
\KeywordTok{print}\NormalTok{(z) }\CommentTok{# mean and SD, the MLE's}
\end{Highlighting}
\end{Shaded}

\begin{verbatim}
##      mean         sd    
##   1.0614409   0.8666937 
##  (0.1891281) (0.1337337)
\end{verbatim}

\begin{Shaded}
\begin{Highlighting}[]
\NormalTok{x <-}\StringTok{ }\DecValTok{0}\OperatorTok{:}\DecValTok{5} \CommentTok{# range of x values across where you want distribution}
\NormalTok{p_density <-}\StringTok{ }\KeywordTok{dnorm}\NormalTok{(}\DataTypeTok{x=}\NormalTok{x,}
                   \DataTypeTok{mean=}\NormalTok{z}\OperatorTok{$}\NormalTok{estimate[}\StringTok{"mean"}\NormalTok{],}
                   \DataTypeTok{sd=}\NormalTok{z}\OperatorTok{$}\NormalTok{estimate[}\StringTok{"sd"}\NormalTok{]) }\CommentTok{# use mean and sd from actual data}
\KeywordTok{qplot}\NormalTok{(x,p_density,}\DataTypeTok{geom=}\StringTok{"line"}\NormalTok{) }\OperatorTok{+}\StringTok{ }\KeywordTok{annotate}\NormalTok{(}\DataTypeTok{geom=}\StringTok{"point"}\NormalTok{,}\DataTypeTok{x=}\NormalTok{aureuslog_data,}\DataTypeTok{y=}\FloatTok{0.0000001}\NormalTok{,}\DataTypeTok{color=}\StringTok{"firebrick3"}\NormalTok{) }\OperatorTok{+}\StringTok{ }\KeywordTok{labs}\NormalTok{(}\DataTypeTok{title=}\StringTok{"S. aureus (Minn.) log10 data against a normal distribution density curve"}\NormalTok{,}
       \DataTypeTok{x=}\StringTok{"BTM S. aureus (Minn.) log10 of 21 herds"}\NormalTok{,}
       \DataTypeTok{y=}\StringTok{"Probability density of normal dist."}\NormalTok{) }\CommentTok{# see data points (x values) fall right in middle of probability mass curve}
\end{Highlighting}
\end{Shaded}

\includegraphics{probability_dist_BTM_40herd_edited_files/figure-latex/unnamed-chunk-22-1.pdf}

\begin{Shaded}
\begin{Highlighting}[]
\CommentTok{# # do the same for the gamma distribution - can't do, has zeroes in it}
\CommentTok{# z <-fitdistr(aureuslog_data,"gamma")}
\CommentTok{# print(z)}
\CommentTok{# p_density <- dgamma(x=x,}
\CommentTok{#                    shape=z$estimate["shape"],}
\CommentTok{#                    rate=z$estimate["rate"])}
\CommentTok{# qplot(x,p_density,geom="line") + annotate(geom="point",x=aureuslog_data,y=0.00000000000001, color="firebrick3") + labs(title="S. aureus (Minn.) log10 data against a gamma distribution density curve",}
\CommentTok{#        x="BTM S. aureus (Minn.) log10 of 21 herds",}
\CommentTok{#        y="Probability density of gamma dist.") #bound at 0}

\KeywordTok{ggdensity}\NormalTok{(btm}\OperatorTok{$}\NormalTok{log10_aureus, }
          \DataTypeTok{main =} \StringTok{"Density plot of BTM S. aureus (Minn) log10"}\NormalTok{,}
          \DataTypeTok{xlab =} \StringTok{"S. aureus counts of 21 herds"}\NormalTok{)}
\end{Highlighting}
\end{Shaded}

\includegraphics{probability_dist_BTM_40herd_edited_files/figure-latex/unnamed-chunk-22-2.pdf}

\begin{Shaded}
\begin{Highlighting}[]
\KeywordTok{ggqqplot}\NormalTok{(btm}\OperatorTok{$}\NormalTok{log10_aureus)}
\end{Highlighting}
\end{Shaded}

\includegraphics{probability_dist_BTM_40herd_edited_files/figure-latex/unnamed-chunk-22-3.pdf}

\subsection{\texorpdfstring{\textbf{\emph{Minnesota Staph.
species}}}{Minnesota Staph. species}}\label{minnesota-staph.-species}

\begin{Shaded}
\begin{Highlighting}[]
\KeywordTok{summary}\NormalTok{(btm}\OperatorTok{$}\NormalTok{Staph_sp)}
\end{Highlighting}
\end{Shaded}

\begin{verbatim}
##    Min. 1st Qu.  Median    Mean 3rd Qu.    Max. 
##    0.00   20.00   65.00   95.48  125.00  665.00
\end{verbatim}

\begin{Shaded}
\begin{Highlighting}[]
\NormalTok{sp_data<-btm}\OperatorTok{$}\NormalTok{Staph_sp}
\NormalTok{sp_data_no0<-sp_data[sp_data}\OperatorTok{>}\DecValTok{0}\NormalTok{]}
\NormalTok{z <-}\KeywordTok{fitdistr}\NormalTok{(sp_data,}\StringTok{"normal"}\NormalTok{)}
\KeywordTok{print}\NormalTok{(z) }\CommentTok{# mean and SD, the MLE's}
\end{Highlighting}
\end{Shaded}

\begin{verbatim}
##      mean         sd    
##    95.47619   136.66003 
##  ( 29.82166) ( 21.08710)
\end{verbatim}

\begin{Shaded}
\begin{Highlighting}[]
\NormalTok{x <-}\StringTok{ }\DecValTok{0}\OperatorTok{:}\DecValTok{800} \CommentTok{# range of x values across where you want distribution}
\NormalTok{p_density <-}\StringTok{ }\KeywordTok{dnorm}\NormalTok{(}\DataTypeTok{x=}\NormalTok{x,}
                   \DataTypeTok{mean=}\NormalTok{z}\OperatorTok{$}\NormalTok{estimate[}\StringTok{"mean"}\NormalTok{],}
                   \DataTypeTok{sd=}\NormalTok{z}\OperatorTok{$}\NormalTok{estimate[}\StringTok{"sd"}\NormalTok{]) }\CommentTok{# use mean and sd from actual data}
\KeywordTok{qplot}\NormalTok{(x,p_density,}\DataTypeTok{geom=}\StringTok{"line"}\NormalTok{) }\OperatorTok{+}\StringTok{ }\KeywordTok{annotate}\NormalTok{(}\DataTypeTok{geom=}\StringTok{"point"}\NormalTok{,}\DataTypeTok{x=}\NormalTok{sp_data,}\DataTypeTok{y=}\FloatTok{0.0000001}\NormalTok{,}\DataTypeTok{color=}\StringTok{"firebrick4"}\NormalTok{) }\OperatorTok{+}\StringTok{ }\KeywordTok{labs}\NormalTok{(}\DataTypeTok{title=}\StringTok{"Staph sp. (Minn.) data against a normal distribution density curve"}\NormalTok{,}
       \DataTypeTok{x=}\StringTok{"BTM Staph. sp. (Minn) of 21 herds"}\NormalTok{,}
       \DataTypeTok{y=}\StringTok{"Probability density of normal dist."}\NormalTok{) }\CommentTok{# see data points (x values) fall right in middle of probability mass curve}
\end{Highlighting}
\end{Shaded}

\includegraphics{probability_dist_BTM_40herd_edited_files/figure-latex/unnamed-chunk-23-1.pdf}

\begin{Shaded}
\begin{Highlighting}[]
\CommentTok{# do the same for the gamma distribution, took out 0 value}
\NormalTok{z <-}\KeywordTok{fitdistr}\NormalTok{(sp_data_no0,}\StringTok{"gamma"}\NormalTok{)}
\KeywordTok{print}\NormalTok{(z)}
\end{Highlighting}
\end{Shaded}

\begin{verbatim}
##       shape         rate    
##   0.998528451   0.009960363 
##  (0.276553909) (0.003521411)
\end{verbatim}

\begin{Shaded}
\begin{Highlighting}[]
\NormalTok{p_density <-}\StringTok{ }\KeywordTok{dgamma}\NormalTok{(}\DataTypeTok{x=}\NormalTok{x,}
                   \DataTypeTok{shape=}\NormalTok{z}\OperatorTok{$}\NormalTok{estimate[}\StringTok{"shape"}\NormalTok{],}
                   \DataTypeTok{rate=}\NormalTok{z}\OperatorTok{$}\NormalTok{estimate[}\StringTok{"rate"}\NormalTok{])}
\KeywordTok{qplot}\NormalTok{(x,p_density,}\DataTypeTok{geom=}\StringTok{"line"}\NormalTok{) }\OperatorTok{+}\StringTok{ }\KeywordTok{annotate}\NormalTok{(}\DataTypeTok{geom=}\StringTok{"point"}\NormalTok{,}\DataTypeTok{x=}\NormalTok{sp_data_no0,}\DataTypeTok{y=}\FloatTok{0.0000001}\NormalTok{, }\DataTypeTok{color=}\StringTok{"firebrick4"}\NormalTok{) }\OperatorTok{+}\StringTok{ }\KeywordTok{labs}\NormalTok{(}\DataTypeTok{title=}\StringTok{"Staph sp. (Minn.) data against a gamma distribution density curve"}\NormalTok{,}
       \DataTypeTok{x=}\StringTok{"BTM Staph sp. (Minn.)  of 21 herds"}\NormalTok{,}
       \DataTypeTok{y=}\StringTok{"Probability density of gamma dist."}\NormalTok{) }\CommentTok{#bound at 0}
\end{Highlighting}
\end{Shaded}

\includegraphics{probability_dist_BTM_40herd_edited_files/figure-latex/unnamed-chunk-23-2.pdf}

\begin{Shaded}
\begin{Highlighting}[]
\KeywordTok{ggdensity}\NormalTok{(btm}\OperatorTok{$}\NormalTok{Staph_sp, }
          \DataTypeTok{main =} \StringTok{"Density plot of BTM Staph. species (Minn)"}\NormalTok{,}
          \DataTypeTok{xlab =} \StringTok{"Staph. species counts of 21 herds"}\NormalTok{)}
\end{Highlighting}
\end{Shaded}

\includegraphics{probability_dist_BTM_40herd_edited_files/figure-latex/unnamed-chunk-23-3.pdf}

\begin{Shaded}
\begin{Highlighting}[]
\KeywordTok{ggqqplot}\NormalTok{(btm}\OperatorTok{$}\NormalTok{Staph_sp)}
\end{Highlighting}
\end{Shaded}

\includegraphics{probability_dist_BTM_40herd_edited_files/figure-latex/unnamed-chunk-23-4.pdf}

\begin{Shaded}
\begin{Highlighting}[]
\CommentTok{#---------------------------------------------------}

\CommentTok{# remove outlier}
\NormalTok{sp_data_trim<-sa_data[sa_data}\OperatorTok{<}\DecValTok{200}\NormalTok{]}
\NormalTok{z <-}\KeywordTok{fitdistr}\NormalTok{(sp_data_trim,}\StringTok{"normal"}\NormalTok{)}
\KeywordTok{print}\NormalTok{(z) }\CommentTok{# mean and SD, the MLE's}
\end{Highlighting}
\end{Shaded}

\begin{verbatim}
##      mean         sd    
##   29.750000   29.558205 
##  ( 6.609416) ( 4.673563)
\end{verbatim}

\begin{Shaded}
\begin{Highlighting}[]
\NormalTok{x <-}\StringTok{ }\DecValTok{0}\OperatorTok{:}\DecValTok{200} \CommentTok{# range of x values across where you want distribution}
\NormalTok{p_density <-}\StringTok{ }\KeywordTok{dnorm}\NormalTok{(}\DataTypeTok{x=}\NormalTok{x,}
                   \DataTypeTok{mean=}\NormalTok{z}\OperatorTok{$}\NormalTok{estimate[}\StringTok{"mean"}\NormalTok{],}
                   \DataTypeTok{sd=}\NormalTok{z}\OperatorTok{$}\NormalTok{estimate[}\StringTok{"sd"}\NormalTok{]) }\CommentTok{# use mean and sd from actual data}
\KeywordTok{qplot}\NormalTok{(x,p_density,}\DataTypeTok{geom=}\StringTok{"line"}\NormalTok{) }\OperatorTok{+}\StringTok{ }\KeywordTok{annotate}\NormalTok{(}\DataTypeTok{geom=}\StringTok{"point"}\NormalTok{,}\DataTypeTok{x=}\NormalTok{sp_data_trim,}\DataTypeTok{y=}\FloatTok{0.0000001}\NormalTok{,}\DataTypeTok{color=}\StringTok{"firebrick4"}\NormalTok{) }\OperatorTok{+}\StringTok{ }\KeywordTok{labs}\NormalTok{(}\DataTypeTok{title=}\StringTok{"Staph sp. (Minn.) data against a normal distribution density curve"}\NormalTok{,}
       \DataTypeTok{x=}\StringTok{"BTM Staph. sp. (Minn) of 21 herds"}\NormalTok{,}
       \DataTypeTok{y=}\StringTok{"Probability density of normal dist."}\NormalTok{) }\CommentTok{# see data points (x values) fall right in middle of probability mass curve}
\end{Highlighting}
\end{Shaded}

\includegraphics{probability_dist_BTM_40herd_edited_files/figure-latex/unnamed-chunk-23-5.pdf}

\begin{Shaded}
\begin{Highlighting}[]
\KeywordTok{ggqqplot}\NormalTok{(sp_data_trim)}
\end{Highlighting}
\end{Shaded}

\includegraphics{probability_dist_BTM_40herd_edited_files/figure-latex/unnamed-chunk-23-6.pdf}

\subsection{\texorpdfstring{\textbf{\emph{Minnesota Staph. species log10
transformation}}}{Minnesota Staph. species log10 transformation}}\label{minnesota-staph.-species-log10-transformation}

\begin{Shaded}
\begin{Highlighting}[]
\KeywordTok{summary}\NormalTok{(btm}\OperatorTok{$}\NormalTok{log10_staphsp)}
\end{Highlighting}
\end{Shaded}

\begin{verbatim}
##    Min. 1st Qu.  Median    Mean 3rd Qu.    Max. 
##   0.000   1.301   1.813   1.667   2.097   2.823
\end{verbatim}

\begin{Shaded}
\begin{Highlighting}[]
\NormalTok{staphsplog_data<-btm}\OperatorTok{$}\NormalTok{log10_staphsp}
\NormalTok{z <-}\KeywordTok{fitdistr}\NormalTok{(staphsplog_data,}\StringTok{"normal"}\NormalTok{)}
\KeywordTok{print}\NormalTok{(z) }\CommentTok{# mean and SD, the MLE's}
\end{Highlighting}
\end{Shaded}

\begin{verbatim}
##       mean          sd    
##   1.66665641   0.59304206 
##  (0.12941239) (0.09150838)
\end{verbatim}

\begin{Shaded}
\begin{Highlighting}[]
\NormalTok{x <-}\StringTok{ }\DecValTok{0}\OperatorTok{:}\DecValTok{5} \CommentTok{# range of x values across where you want distribution}
\NormalTok{p_density <-}\StringTok{ }\KeywordTok{dnorm}\NormalTok{(}\DataTypeTok{x=}\NormalTok{x,}
                   \DataTypeTok{mean=}\NormalTok{z}\OperatorTok{$}\NormalTok{estimate[}\StringTok{"mean"}\NormalTok{],}
                   \DataTypeTok{sd=}\NormalTok{z}\OperatorTok{$}\NormalTok{estimate[}\StringTok{"sd"}\NormalTok{]) }\CommentTok{# use mean and sd from actual data}
\KeywordTok{qplot}\NormalTok{(x,p_density,}\DataTypeTok{geom=}\StringTok{"line"}\NormalTok{) }\OperatorTok{+}\StringTok{ }\KeywordTok{annotate}\NormalTok{(}\DataTypeTok{geom=}\StringTok{"point"}\NormalTok{,}\DataTypeTok{x=}\NormalTok{staphsplog_data,}\DataTypeTok{y=}\FloatTok{0.0000001}\NormalTok{,}\DataTypeTok{color=}\StringTok{"firebrick4"}\NormalTok{) }\OperatorTok{+}\StringTok{ }\KeywordTok{labs}\NormalTok{(}\DataTypeTok{title=}\StringTok{"Staph sp. (Minn.) log10 data against a normal distribution density curve"}\NormalTok{,}
       \DataTypeTok{x=}\StringTok{"BTM Staph. sp (Minn.) log10 of 21 herds"}\NormalTok{,}
       \DataTypeTok{y=}\StringTok{"Probability density of normal dist."}\NormalTok{) }\CommentTok{# see data points (x values) fall right in middle of probability mass curve}
\end{Highlighting}
\end{Shaded}

\includegraphics{probability_dist_BTM_40herd_edited_files/figure-latex/unnamed-chunk-24-1.pdf}

\begin{Shaded}
\begin{Highlighting}[]
\CommentTok{# # do the same for the gamma distribution - can't do, has zeroes in it}
\CommentTok{# z <-fitdistr(staphsplog_data,"gamma")}
\CommentTok{# print(z)}
\CommentTok{# p_density <- dgamma(x=x,}
\CommentTok{#                    shape=z$estimate["shape"],}
\CommentTok{#                    rate=z$estimate["rate"])}
\CommentTok{# qplot(x,p_density,geom="line") + annotate(geom="point",x=staphsplog_data,y=0.00000000000001, color="firebrick3") + labs(title="Staph. sp (Minn.) log10 data against a gamma distribution density curve",}
\CommentTok{#        x="BTM Staph. sp (Minn.) log10 of 21 herds",}
\CommentTok{#        y="Probability density of gamma dist.") #bound at 0}

\KeywordTok{ggdensity}\NormalTok{(btm}\OperatorTok{$}\NormalTok{log10_staphsp, }
          \DataTypeTok{main =} \StringTok{"Density plot of BTM Staph. sp (Minn) log10"}\NormalTok{,}
          \DataTypeTok{xlab =} \StringTok{"Staph. sp counts of 21 herds"}\NormalTok{)}
\end{Highlighting}
\end{Shaded}

\includegraphics{probability_dist_BTM_40herd_edited_files/figure-latex/unnamed-chunk-24-2.pdf}

\begin{Shaded}
\begin{Highlighting}[]
\KeywordTok{ggqqplot}\NormalTok{(btm}\OperatorTok{$}\NormalTok{log10_staphsp)}
\end{Highlighting}
\end{Shaded}

\includegraphics{probability_dist_BTM_40herd_edited_files/figure-latex/unnamed-chunk-24-3.pdf}

\end{document}
